\documentclass[11pt]{amsart}

\usepackage{../macros-master,amsaddr,physics}
\usepackage{mathbbol}

\addbibresource{refs.bib}

\newcommand{\fields}{\cE}
\newcommand{\Vir}{\sV\op{ir}}
\newcommand{\vir}{\lie{vir}}
\setcounter{tocdepth}{1}

\renewcommand{\op}{\operatorname}

%\linespread{1.2} %for editing
%\usepackage{mathpazo}


\begin{document}

\title{A higher dimensional Segal--Sugawara construction}
\author{Brian R. Williams}
\thanks{Boston University, Department of Mathematics and Statistics}
\email{bwill22@bu.edu}
\maketitle

Let $X$ be a three-dimensional complex manifold. Denote by $\Omega^1_X$ the sheaf of holomorphic one-forms and by $\del \cO_X$ the sheaf of exact holomorphic one-forms.

\section{Virasoro factorization algebras}

Let $X$ b
e a complex manifold of dimension $n$.
The sheaf of holomorphic vector fields has the natural structure of a sheaf of Lie algebras with bracket the Lie bracket of vector fields.
Since this bracket involves only holomorphic differential operators, it extends to a bracket on the Dolbeault complex $\Omega^{0,\bu}(X,\T_X)$ of the holomorphic tangent bundle.
This bracket endows the Dolbeault complex $\Omega^{0,\bu}(X, \T_X)$ with coefficients in the holomorphic tangent bundle with the structure of a dg Lie algebra where the differential is $\dbar$.
Its global sections is an explicit model for the dg Lie algebra $\R\Gamma(X , \cT_X)$ where $\cT_X$ is the sheaf of holomorphic vector fields on $X$.

As defined in \cite{CG2}, a local dg Lie algebra is a $\Z$-graded complex of vector bundles whose sheaf of sections is equipped with the structure of a dg Lie algebra.
Let $\cT_X$ be the local dg Lie algebra whose underlying complex of vector bundles is
\begin{equation}
\Omega^{0,\bu}(X, \T_X) = \Gamma_{C^\infty} \left(X , \wedge^\bu \br \T_X \otimes \T_X \right) .
\end{equation}
The differential is the $\dbar$-operator associated to the holomorphic bundle $\T_X$.
The bracket is the Lie bracket of holomorphic vector fields extended to the Dolbeault complex as described above.
By Dolbeault's theorem, for $U \subset X$ a Stein open set there is a quasi-isomorphism $\cT_X(U) \simeq \op{Vect}^{hol}(U)$, the Lie algebra of holomorphic vector fields on~$U$.
When $X = \C^n$ we denote $\cT_X = \cT$ for simplicity.

%There is also the following relationship between $\cT$ and formal vector fields, which will be used subsequently.
%Let \(E \to M\) denote a $\Z$-graded vector bundle on \(M\).
%We consider the pro vector bundle of $\infty$-jets which we will denote by $j^\infty E$, see \cite{Anderson} or \cite[\S 5.6]{CostelloBook} for instance.
%The sheaf of smooth sections of this pro vector bundle carries the natural structure of a $D_M$-module.
%If $\cL = \Gamma(L)$ is a local $L_\infty$ algebra then $j^\infty L$ is a bundle of $L_\infty$ algebras.
%
%\begin{prop}
%The central charge of the
%\end{prop}
%
%\begin{prop}
%Let $X$ be a complex manifold of dimension $n$ and $\mathbf{c} \in H^{2n+2}(BU(n))$.
%Then, there exists a unique factorization algebra $X$ with the property that the restriction of it to any coordinatized open set $U$ is equivalent to the Virasoro factorization algebra $\cV\op{ir}_{\mathbf{c}}$ on $U$.
%\end{prop}
%
%As a consequence, we see that for any such central charge $\mathbf{c} \in H^{2n+2}(BU(n))$ the factorization algebra $\cV \op{ir}_{\mathbf{c}}$ is defined on the entire \textit{site} of complex manifolds of dimension $n$.

The dg Lie algebra $\cT(\C - \{0\})$ is quasi-isomorphic to the Lie algebra of holomorphic vector fields on~$\C^\times$.
Replacing $\C^\times$ by the formal punctured disk gives the Witt algebra $\lie{vect}(D^\times) = \C((z)) \del_z$ whose unique nontrivial central extension is the Virasoro Lie algebra.

For an example of a complex manifold which is not Stein consider $X = \C^n - \{0\}$.
The embedding of dg Lie algebras
\begin{equation}
\Vect^{hol}(\C^n - \{0\}) \hookrightarrow \cT(\C^n - \{0\})
\end{equation}
is no longer a quasi-isomorphism.
This map does define an isomorphism in $H^0$ cohomology, but $\cT(\C^n-\{0\})$ also has cohomology in degree $(n-1)$. 
In standard coordinates, cochain representatives for this cohomology are given by expressions of the form
\begin{equation}
\sum_j \left(\sum_{k_1 \cdots k_n} a^j_{k_1\cdots k_n} L_{\del_{z_1}^{k_1}} \cdots L_{\del_{z_n}^{k_n}} \omega_{BM}\right) \frac{\del}{\del z_j} 
\end{equation}
Here $a^j_{k_1\cdots k_n}$ are constants and $\omega_{BM} \in \Omega^{0,n-1}(\C^n-\{0\})$ is the Bochner--Martinelli form characterized by the formula $\oint_{S^{2n-1}} \omega_{BM} \wedge \d^n z= 1$ with $S^{2n-1}$ any sphere centered at $0 \in \C^n$.

We make the distinction between the complex manifold $\C^n-\{0\}$, the algebraic variety of punctured affine space $\mathring{\A}^{n}=\A^n - \{0\}$, and the punctured formal disk $\mathring{D}^{n}=D^{n}-\{0\}$.
A model for the derived algebra of functions for the complex manifold $\C^{n}-\{0\}$ is the Dolbeault complex $\Omega^{0,\bu}(\C^{n}-\{0\})$.
Using the Jouanolou torsor, an explicit model for the derived algebra of functions on punctured affine space (respectively, formal disk) has been constructed in \cite{FHK} and we will denote it $\sfA_{[n]}$ (respectively, $\sfA_{n}$).
It is an algebraic (respectively, formal) version of the Dolbeault complex of the complex manifold $\C^{n}-\{0\}$.
Similarly, there is an algebraic (respectively, formal) version of $\cT(\C^n - \{0\})$ with $\C^n - \{0\}$ replaced by punctured $n$-dimensional algebraic affine space $\mathring{\A}^{n}$ (respectively, the punctured formal disk $\mathring{D}^{n}$).
In the affine (respectively, formal) case these models are cochain models for the derived global sections of the tangent sheaf $\R \Gamma(\mathring{\A}^{n},\T)$ (respectively, $\R \Gamma(\mathring{D}^{n},\T)$).

Using a dg model for the punctured $n$-space presented in \cite{FHK}, in appendix \ref{appx:A} we construct a dg model for the derived global sections of the tangent bundle over the punctured formal disk which we will denote by $\lie{witt}(n)$ and refer to as the \defterm{$n$-dimensional Witt algebra}.
Here is a list of important properties of the dg Lie algebra $\lie{witt}(n)$:
\begin{itemize}
  \item It is a dg model for $\R \Gamma(\mathring{D}^{n},\T)$.
        The underlying cochain complex $\lie{witt}(n)$ is concentrated in cohomological degrees $0,\ldots,n-1$. In particular, its cohomology is concentrated in degrees $0$ and $n-1$.
  \item We have the following local trivializations.
        Any element of $\xi \in \lie{witt}(n)$ is of the form
        \begin{equation}\label{}
          \xi = \sum_{i=1}^{n} \alpha_{i}(z) \frac{\del}{\del z_{i}}
        \end{equation}
        where $\alpha_{i}(x) \in \sfA_{n}$.
\item There is an inclusion of dg Lie algebras from vector fields on the formal disk $D^n$ (thought of as a dg Lie algebra concentrated in degree zero):
\begin{equation}
\lie{vect}(n) = \op{Der} \, \C[[z_1,\ldots,z_n]] \hookrightarrow
\lie{witt}(n)
\end{equation}
which induces an isomorphism in zeroth cohomology.
\item In degree $(n-1)$ \brian{finish}
\end{itemize}

The Virasoro Lie algebra is the unique nontrivial central extension of the ordinary Witt algebra $\lie{witt}(1)$.
This central extension admits an interpretation in terms of the Lie algebra cohomology of  vector fields $\lie{vect}(1)$ on the $1$-disk via the isomorphisms
\begin{equation}
H^3(\lie{vect}(1); \C) \cong H^2(\lie{vect}(1); \Omega^1) \cong H^2(\lie{witt}(1);\C) \cong \C .
\end{equation}
This is a version of transgression, and in this example it can be understood explicitly as follows.
Consider the dg $\lie{vect}(1)$-module $\Omega^{\bu}$ defined by
\begin{equation}\label{}
  \cO(D) = \C[[z]] \xto{\del} \C[[z]] \d z = \Omega^{1}(D) [-1]
\end{equation}
where the differential is the de Rham differential.
By the formal Poincar\'{e} lemma, there is a quasi-isomorphism of $\lie{vect}(1)$-modules $\C \xto{\simeq} \Omega^{\bu}$ and a map \brian{finish}

Take a class $[\phi] \in H^{3}(\lie{vect}(1))$.


In the last part of this section we will construct a linear map
\begin{equation}
H^{2n+1}(\lie{vect}(n);\C) \to H^2(\lie{witt}(n);\C), 
\end{equation}
thus providing a source of central extensions for the $n$-dimensional Witt algebra.
Our construction of central extensions uses factorization algebras.

\subsection{Local cohomology of vector fields}

Such central extensions originate from the so-called local cohomology of holomorphic vector fields.
We briefly recall the definition of local cohomology from \cite{CG2}.
For details specific to the local Lie algebra of holomorphic vector fields we refer to \cite{BWgf}.

For $E$ a vector bundle on a manifold $M$, define the space of local functionals to be 
\begin{equation}
\oloc(E) = {\rm Dens}_M \otimes_{D_M} {\rm Lag} (E) .
\end{equation}
Here $\op{Dens}_X$ is the bundle of densities on $M$ and $\op{Lag}(E)$ is the left $D_M$-module of Lagrangians
\begin{equation}
\op{Lag} (E) \define \prod_{n > 0} {\rm Hom}_{C^\infty_M} \left( \Sym^n(J^\infty E) , C^\infty_M\right) .
\end{equation}
Concretely, a Lagrangian is a $C^\infty_M$-valued functional which only depends on jets of sections of $E$, and a local functional is a Lagrangian defined up to total derivatives.
Notice that we throw away the constant coefficient Lagrangian densitites.
Given a Lagrangian $L = L(\phi)$ we will denote its corresponding local functional by~$\int L(\phi)$.

If $E$ is a complex of vector bundles then $\oloc(E)$ inherits the differential and is a sheaf of cochain complexes.
If $\cL = \Gamma(M, L)$ is a local Lie algebra then $\oloc(L[1])$ is equipped with the Chevalley--Eilenberg differential, hence giving a local version of Lie algebra cohomology.
We denote this sheaf of cochain complexes by $C^\bu_{loc}(\cL)$.
Note that for every open set $U \subset M$ there is an embedding map
\begin{equation}\label{eqn:localtocompact}
C^\bu_{loc}(\cL)(U) \hookrightarrow C^\bu(\cL_c(U)) 
\end{equation}
where the right hand side is the Chevalley--Eilenberg complex computing the Lie algebra cohomology of compactly supported sections of $L$ on $U$.
The map sends $\int \cL|_{U}$ to the honest integral $\int_{U} \cL$.
Notice that the latter expression is defined by the condition of compact support.

The local cohomology of $\cL=\cT_X$, the local Lie algebra of holomorphic vector fields on a complex manifold $X$, has been characterized in \cite{BWgf}.
In the theorem below,~$Fr_X$ denotes the principal $U(n)$-bundle of unitary frames on $X$.

\begin{thm}[\cite{BWgf}]
Let $Y_n$ be the restriction of the universal principal $U(n)$-bundle to the $2n$-skeleton of $BU(n)$.
Then
\begin{equation}
H^\bu_{loc} (\cT(X)) \cong H^\bu(Fr_X \times^{U(n)} Y_n)[2n] .
\end{equation}

In particular, when $X = \C^n$ there are isomorphisms
\begin{equation}
H^\bu_{loc} (\cT(\C^n)) \cong H^\bu(\lie{vect}(n) ; \C)[2n] \cong H^\bu(Y_n)[2n] ,
\end{equation}
where $H^\bu(\lie{vect}(n) ; \C)$ is the Gelfand--Fuks Lie algebra cohomology of $\lie{vect}(n)$ with trivial coefficients.
\end{thm}

We are most interested in the degree one local cohomology.
A simple application of the Serre spectral sequence reveals that in this case 
\begin{equation}
H^1_{loc}(\cT(\C^n)) \simeq H^{2n+1}(\lie{vect}(n)) \simeq H^{2n+2} (BU(n)) .
\end{equation}
So, degree one local cohomology classes for $\cT$ are in one-to-one correspondence with degree $2n+2$ polynomials in the universal Chern classes $c_1,\ldots,c_n$.


\subsection{Higher Virasoro Lie algebras}

At this point, we can see how central extensions of the $n$-dimensional Witt algebra appear.
The morphism \eqref{eqn:localtocompact} shows how local cohomology classes of a local Lie algebra $\cL$ supported on $U$ determine Lie algebra cohomology classes for the Lie algebra of compactly supported sections $\cL_c(U)$.
We showed in appendix \ref{appx:A} that the $n$-dimensional Witt algebra $\lie{witt}(n)$ admits a homomorphism of dg Lie algebras
\begin{equation}
j \colon \lie{witt}(n) \hookrightarrow \Omega^{0,\bu} (\C^n - \{0\}, \T) = \cT(\C^n - \{0\}),
\end{equation}
which has the property that in cohomology $H^\bu(j)$ is a dense embedding of topological vector spaces.

We consider the local dg Lie algebra $\Omega^\bu(\R) \otimes \lie{witt}(n)$ defined on the manifold $\R$.
The differential is 
\begin{equation}
\d_{dR} \otimes \id + \id \otimes \dbar_{\lie{witt}},
\end{equation}
where the first term is the de Rham operator on $\R$ and the second is the internal differential to the dg Lie algebra $\lie{witt}(n)$.
The bracket is
\begin{equation}\label{eqn:tensorbracket}
[\eta \otimes X, \theta \otimes Y] = (\eta \theta) \otimes [X,Y] , \quad \eta,\theta \in \Omega^\bu(\R), \;\; X,Y \in \lie{witt}(n)
\end{equation}
where $[-,-]$ is the Lie bracket in $\lie{witt}(n)$.
Using $j$ we now construct a map (up to homotopy) of the form
\begin{equation}\label{eqn:tilj}
\til j \colon \Omega^\bu_c (\R) \otimes \lie{witt}(n) \to \cT_c (\C^n - 0) .
\end{equation}

Introduce the following local dg Lie algebra $\cL_{log}$ on $\R$ which is the same underlying complex of vector bundles as $\Omega^\bu (\R) \otimes  \lie{witt}(n)$, except with the Lie bracket defined by the formulas
\begin{align*}
\left[f \otimes \alpha \frac{\del}{\del z_i} , g \otimes \beta \frac{\del}{\del z_j}\right]_{\cL_{log}} & = fg \otimes \left[\alpha \frac{\del}{\del z_i}, \beta \frac{\del}{\del z_j}\right] +  f g' \otimes \lambda_i \alpha \beta \frac{\del}{\del z_j} - (-1)^{\alpha \beta} f' g \otimes \lambda_j \alpha \beta  \frac{\del}{\del z_i}  \\
\left[f \otimes \alpha \frac{\del}{\del z_i} , g \d t \otimes \beta \frac{\del}{\del z_j}\right]_{\cL_{log}} & = fg \d t \otimes \left[\alpha \frac{\del}{\del z_i}, \beta \frac{\del}{\del z_j}\right] + f g' \d t \otimes \lambda_i \alpha \beta \frac{\del}{\del z_j} - (-1)^{\alpha \beta} f' g \d t \otimes \lambda_j \alpha \beta  \frac{\del}{\del z_i} \\
& + fg \otimes \alpha (\dbar \lambda_i) \beta \frac{\del}{\del z_j} , \\ 
\left[f \d t \otimes \alpha \frac{\del}{\del z_i} , g \d t \otimes \beta \frac{\del}{\del z_j}\right]_{\cL_{log}} & = 0 .
\end{align*}
Here $\alpha, \beta \in \sfA_n$, $f,g \in C^\infty(\R)$, and we have used the notation from the appendix $\lambda_i = \frac{\zbar_i}{z \zbar}$.\footnote{When $n=1$ we set $\lambda = \frac{1}{z}$, and the brackets in $\cL_{log}$ simplify to
\[
[f \otimes L_n, g \otimes L_m ] = (m-n) fg \otimes L_{n+m} + (fg' - f'g) \otimes L_{n+m} ,
\] 
for example.  }
Notice that the first terms on the right hand sides of the expressions above agree with the Lie bracket in \eqref{eqn:tensorbracket}.
We also remark that the second line of the second equation is identically zero when $n = 1$.

We now construct an $L_\infty$ equivalence 
\begin{equation}
\Phi = (\Phi^{(1)}, \Phi^{(2)}) \colon \Omega^\bu(\R) \otimes \lie{witt}(n) \rightsquigarrow \cL_{log} .
\end{equation}
Let $\Phi^{(1)} = \id$ and 
\begin{align*}
\Phi^{(2)}\left(f \d t\otimes \alpha \frac{\del}{\del z_i} , g \otimes \beta \frac{\del}{\del z_j} \right) & = fg \otimes \lambda_i \alpha \beta \frac{\del}{\del z_j} \\
\Phi^{(2)}\left(f \d t \otimes \alpha \frac{\del}{\del z_i} , g \d t \otimes \beta \frac{\del}{\del z_j} \right) & = fg \d t \otimes \alpha \beta \left(\lambda_i \frac{\del}{\del z_j} + \lambda_j \frac{\del}{\del z_i} \right) .
\end{align*}
To see that $\Phi = (\Phi^{(1)}, \Phi^{(2)})$ is an $L_\infty$ morphism we observe the following relations:
\begin{multline}
\Phi^{(2)}\left(\d f \otimes \alpha \frac{\del}{\del z_i} , g \otimes \beta \frac{\del}{\del z_j}\right) - (-1)^{\alpha} \Phi^{(2)}\left(f \otimes \alpha \frac{\del}{\del z_i} , \d g \otimes \beta \frac{\del}{\del z_j}\right) \\ = f g' \otimes \lambda_i \alpha \beta \frac{\del}{\del z_j} - (-1)^{\alpha \beta} f' g \otimes \lambda_j \alpha \beta  \frac{\del}{\del z_i} ,
\end{multline}
and
\begin{multline}
- (\d + \dbar_{\lie{witt}}) \Phi^{(2)}\left(f \otimes \alpha \frac{\del}{\del z_i} , g \d t \otimes \beta \frac{\del}{\del z_j}\right) + \Phi^{(2)}\left(\d f \otimes \alpha \frac{\del}{\del z_i} , g \d t \otimes \beta \frac{\del}{\del z_j}\right) \\ + \Phi^{(2)}\left(f \otimes \dbar \alpha \frac{\del}{\del z_i} , g \d t \otimes \beta \frac{\del}{\del z_j}\right) - (-1)^\alpha \Phi^{(2)}\left(f \otimes \alpha \frac{\del}{\del z_i} , g \d t \otimes \dbar \beta \frac{\del}{\del z_j}\right)  \\ = f g' \d t \otimes \lambda_i \alpha \beta \frac{\del}{\del z_j} - (-1)^{\alpha \beta} f' g \d t \otimes \lambda_j \alpha \beta  \frac{\del}{\del z_i} + fg \otimes \alpha (\dbar \lambda_i) \beta \frac{\del}{\del z_j} ,
\end{multline}
where $f,g \in C^\infty(\R)$ and $\alpha \frac{\del}{\del z_i}, \beta \frac{\del}{\del z_j} \in \lie{witt}(n)$.

The $L_\infty$-map $\Phi$ clearly is given by differential operators, so it defines a homomorphism of \textit{local} dg Lie algebras on the manifold $\R$.
We have shown the following.

\begin{prop}
The local dg Lie algebras $\Omega^\bu_\R \otimes \lie{witt}(n)$ and $\cL_{log}$ are $L_\infty$-equivalent.
In particular, the map $\Phi$ defines a quasi-isomorphism of commutative dg algebras
\begin{equation}
\Phi^* \colon C^\bu(\cL_{log,c}(\R)) \xto{\simeq} C^\bu(\Omega^{\bu}_c(\R) \otimes \lie{witt}(n))
\end{equation}
which restricts to a quasi-isomorphism
\begin{equation}
\Phi^* \colon C_{loc}^\bu(\cL_{log}) \xto{\simeq} C_{loc}^\bu(\Omega^\bu(\R) \otimes \lie{witt}(n)) .
\end{equation}
\end{prop}

Next we define a homomorphism of dg Lie algebras
\begin{equation}
\til j \colon \cL_{log}(\R) \to \cT(\C^n - \{0\})
\end{equation}
by the formulas
\begin{align*}
\til j \left(f(t) \otimes \alpha \frac{\del}{\del z_i}\right) & = f(\log z \zbar) \alpha \frac{\del}{\del z_i} \\
\til j \left(f(t) \d t \otimes \alpha \frac{\del}{\del z_i}\right) & = f(\log z \zbar) \dbar (\log z \zbar )\alpha \frac{\del}{\del z_i} .
\end{align*}
It is immediate to verify that this is a cochain map and it is compatible with Lie brackets. (In fact, the complicated form of the Lie bracket defining $\cL_{log}$ was motivated by this embedding.)

\begin{lem}
The cochain map
\begin{equation}
\til j^* \colon C^\bu (\cT_c(\C^n - 0)) \to C^\bu \left(\cL_{log,c}(\R) \right)
\end{equation}
given by restriction along $\til j$ preserves local cochains. 
Furthermore, if $\phi$ is a local cochain for $\cT(\C^n - 0)$ then $\til j^* \phi$ is translation invariant as a local cochain on $\R$. 
Thus, $\til j$ determines a homomorphism
\begin{equation}
\til j^* \colon C^\bu_{loc}(\cT(\C^n-0)) \to C^\bu_{loc}\left(\cL_{log}\right)^{\R} .
\end{equation}
\end{lem}

Finally, we have the following relationship between translation invariant local cohomology classes for the local Lie algebra $\Omega^\bu \otimes \lie{witt}(n)$ and Lie algebra cohomology classes for $\lie{witt}(n)$.

\begin{lem}
Let $\lie{g}$ be any dg Lie algebra and consider the resulting local Lie algebra $\Omega^\bu_\R \otimes \lie{g}$ on the manifold $\R$.
Then
\begin{equation}
H_{loc}^\bu\left(\Omega^\bu_\R \otimes \lie{g}\right)^\R \cong H^{\bu+1}(\lie{g}) .
\end{equation}
\end{lem}
\begin{proof}
Let $E$ be a vector bundle on $\R^n$.
In \cite[??]{CostelloBook} it is shown that a choice of a flat connection on $E$ produces a quasi- isomorphism 
\begin{equation}
\oloc(E)^{\R^n} \simeq \left(\cO(J_0 E) \slash \C\right) \otimes^{\mathbf{L}}_{\R[\del_1,\ldots,\del_n]} |\det|_{\R^n} \end{equation}
where $\del_i$ are the coordinate vector fields and $J_0 E$ denotes the fiber of the $\infty$-jet bundle of $E$ at the origin $0 \in \R^n$.
We apply this to the case that $E[1] = L$ is a local Lie algebra, where we equip local functionals with the Chevalley--Eilenberg differential.
Then, this result says that
\begin{equation}
C^\bu_{loc}(\cL)^{\R^n} \simeq C^\bu_{red}(J_0 L) \otimes^{\mathbb{L}}_{\R[\del_1,\ldots,\del_n]} |\det|_{\R^n} 
\end{equation}
where $J_0 L$ is the topological Lie algebra obtained by taking the $\infty$-jets of $L$ at $0 \in \R^n$, and is equipped with the bracket obtained by Taylor expanding the bracket on $L$.

We apply this to the case $\cL = \Omega^\bu_\R \otimes \lie{g}$.
Then, by the formal Poincar\'e lemma
\begin{equation}
(J_0 L, \d_{dR}) \simeq \left(\C[[x,\d x]] \otimes \lie{g}, \d_{dR} \otimes \id\right) \simeq \left(\lie{g}, \d = 0\right) .
\end{equation}
This implies the result.
\end{proof}

Combining all of these results, we see that the composition
\begin{equation}
C_{loc}^\bu (\cT) \xto{\til j^*} C^\bu_{loc}(\cL_{log})^{\R} \xto{\Phi^*} C^\bu_{loc}(\Omega^\bu_\R \otimes \lie{witt}(n))^{\R} 
\end{equation}
defines, at the level of degree one cohomology, a linear map
\begin{equation}
H^{2n+2}(BU(n)) \to H^{2} (\lie{witt}(n)) .
\end{equation}

\begin{dfn}
Let $\bc \in H^{2n+2}(BU(n))$ be a degree $2n+2$ polynomial in the universal Chern classes $c_1,\ldots,c_n$.
Define $\lie{vir}_{\bc}(n)$ to be the central extension of dg Lie algebras
\begin{equation}
0 \to \C \to \lie{vir}_{\bc}(n) \to \lie{witt}(n) \to 0  \quad .
\end{equation}
We refer to this as the $n$-dimensional Virasoro Lie algebra of \defterm{central charge} $\bc$.
\end{dfn}

We can be explicit about the correspondence between universal characteristic classes and central extensions of $\lie{witt}(n)$ for small values of $n$.
The Lie algebra $\lie{witt}(1)$ is the ordinary Witt algebra consisting of expressions $f(z) \del_z$ where $f(z) \in \C((z))$.
Following \cite{BWgf}, the class $c_1^2 \in H^4(BU(1))$ is proportional to the local cocycle
\begin{equation}
\int_{\C} J \mu \del J \mu
\end{equation}
where $\mu = \alpha (z,\zbar) \del_z$ is a section of $\cT$ and $J \mu = \del_z \alpha(z,\zbar)$.
Following the above restrictions and equivalences, we see that this corresponds to the usual Virasoro cocycle
\begin{equation}
(f \del_z, g \del_z) \mapsto \op{Res}_z f'(z) g''(z) \d z , \quad f,g \in \C((z))
\end{equation}
Of course, in this case, all Virasoro Lie algebras corresponding to a nonzero central charge are isomorphic.

Next consider $n=2$, so that the possible central charges are linear combinations of $c_1^3$ and $c_1 c_2 \in H^4(BU(2))$.
Consider first the class $c_1^3$.
It is shown in \cite{BWgf} that a local cocycle representative corresponding to this class is proportional to
\begin{equation}
\int_{\C^2} \op{Tr}(J \mu) \del \op{Tr}(J \mu) \del \op{Tr}(J \mu) .
\end{equation}
The corresponding degree two cocycle of $\lie{witt}(2)$ is
\begin{equation}
c_1^3 \colon (\mu_0,\mu_1,\mu_2) \mapsto \op{Res}_z \op{Tr}(J \mu_0) \del \op{Tr}(J \mu_1) \del \op{Tr}(J \mu_2)
\end{equation}
where $\mu_i \in \lie{witt}(2)$.
Here $\op{Res}_z \colon \sfA_2 \to \C[-1]$ is the higher residue as defined in appendix \ref{appx:model}, see also \cite{FHK}.
We observe that sense the residue carries degree one, and that the cocycle above is of polynomial degree three, that $c_1^3$ of $\lie{witt}(2)$ is of total degree two as desired.
The other class of two-dimensional Virasoro algebras are of central charge proportional to $c_1\op{ch}_2 = \frac12 (c_1^3 - 2 c_1 c_2)$.
This central extension is represented by the three-linear cocycle
\begin{equation}
c_1 \ch_2 \colon (\mu_0,\mu_1,\mu_2) \mapsto \op{Res}_z \op{Tr}(J \mu_0) \op{Tr}(\del J \mu_1 \del J \mu_2) + \cdots ,
\end{equation}
where the $\cdots$ denote appropriate anti-symmetrization.

\subsection{Enveloping factorization algebras}
\label{s:enveloping}

We recall the construction of the enveloping factorization algebra from \cite{CG2}.
Let $\cE$ be a vector bundle on a manifold $M$ and let $\cE_c$ denote its cosheaf of compactly supported sections.
It is shown in \cite{CG1} how the assignment of the graded vector space
\begin{equation}
\Sym \, \cE_c(U) = \oplus_{n \geq 0} \Gamma_c(U, \Sym^n E)
\end{equation}
to an open set $U \subset M$ assemble into a factorization algebra on $M$.

Now, let $\cL = \Gamma(M,L)$ be a local dg Lie algebra on a manifold $M$.
Applying the above construction to $E[-1] = L$ gives rise \cite{CG2} to the factorization algebra which assigns to an open set $U \subset M$ the graded vector space 
\begin{equation}
\Sym \, \cL_c (U) [1] .
\end{equation}
For each open set $U$ this graded vector space is equipped with a differential $\d_{CE}$ whose cohomology computes the Lie algebra homology of the dg Lie algebra $\cL_c(U)$.
The resulting dg factorization algebra is denoted by $C_\bu(\cL_c)$ and is called the \defterm{enveloping factorization algebra} associated to the local Lie algebra $\cL$. 

Next, we recall shifted central extensions of dg Lie algebras.
Suppose that $(\lie{g},\d_{\lie{g}})$ is a dg Lie algebra and consider its Chevalley--Eilenberg complex $C^\bu(\lie{g};\C)$ computing its Lie algebra cohomology.
This complex is bigraded with the first grading being the internal degree to $\lie{g}$ and the second grading being the graded polynomial degree.
Suppose $\phi$ is a cochain of totalized degree $k$; we assume that it is concentrated in graded polynomial degree $\geq 1$.
Then $\phi$ can be expressed as
\begin{equation}
\phi = \phi_1 + \phi_2 + \cdots 
\end{equation}
where $\phi_i$ carries polynomial degree $i$ and internal cohomological degree $k-i$.

Suppose now that $\phi$ is a cocycle, meaning it is closed for the totalized Chevalley--Eilenberg differential.
It then defines an $L_\infty$-algebra central extension
\begin{equation}
0 \to \C K [k-2] \to  \til{\lie{g}} \to \lie{g} \to 0
\end{equation}
with $\ell$-ary brackets defined by
\begin{itemize}
\item $\ell = 1$
The differential is $[-]_1 = \d_{\lie{g}} + K \phi_1$ where $K \phi_1 \colon \lie{g} \to \C K$ is the first term in the expansion of $\phi$.
\item $\ell=2$. 
The bracket is $[x,y] = [x,y]_{\lie{g}} + K \phi_2(x,y)$, where $x,y \in \lie{g}$.
\item $\ell > 2$.
The $\ell$-ary bracket is
\begin{equation}
[x_1,\ldots,x_\ell]_\ell = K \phi_\ell(x_1,\ldots,x_\ell) .
\end{equation}
\end{itemize}
The central summand of $\til{\lie{g}}$ appears in cohomological degree $2-k$. 
The Chevalley--Eilenberg complex computing Lie algebra homology $C_\bu(\til{\lie{g}})$ is a dg $\C[K]$-module where~$K$ carries cohomological degree $k-1$.


For any local Lie algebra $\cL$ on $M$ we recalled how to construct its enveloping factorization algebra $C_\bu(\cL_c)$ on $M$.
Given a local cocycle $\phi$ of cohomological degree $+1$, there is a $\phi$-twisted version of the enveloping factorization algebra $C_\bu^\phi(\cL_c)$ which we will now recall. 
For more details we refer to \cite[\S 11]{CG2}.

For every open set $U \subset M$ there is a cochain map
\begin{equation}
C^\bu_{loc}(\cL)(U) \to C^\bu(\cL_c(U))
\end{equation}
So local cocycles define cocycles for the cosheaf of dg Lie algebras $U \mapsto \cL_c(U)$.
In particular, $\phi$ defines a precosheaf of $L_\infty$ algebras
\begin{equation}
0 \to \ul \C[-1] \to \til \cL_c \to \cL_c \to 0 ,
\end{equation}
where $\ul \C[-1]$ is the constant precosheaf which assigns $\C[-1]$ in degree $+1$ to every open set.
In \cite{CG2} it is shown that the assignment
\begin{equation}
U \subset M \mapsto C_\bu(\til \cL_c (U)) 
\end{equation}
defines a factorization algebra on $M$ in $\C[K]$-modules (where, now $K$ has degree zero).
This is the \defterm{$\phi$-twisted} factorization enveloping algebra which we will denote by $C^\phi_\bu(\cL_c)$.
It has the property that the $K=0$ specialization is the (untwisted) enveloping factorization algebra $C_\bu(\cL_c)$.

\subsection{Virasoro factorization algebras}

\begin{dfn}\label{dfn:vir}
Fix a complex dimension $n$ and let
\begin{equation}
\bc \in H^{2n+2}(BU(n)) \cong H^1_{loc}(\cT_{\C^n}) .
\end{equation}
The \defterm{Virasoro factorization algebra} on $\C^n$ is the twisted enveloping factorization algebra of $\cT_{\C^n}$ corresponding to this local cohomology class
\begin{equation}
\Vir_{\bc} \define C_\bu^{\bc}(\cT_{\C^n,c})|_{K=1} .
\end{equation}
\end{dfn}

Some motivation for this terminology is the result of \cite{BWvir} which exhibits the relationship between the ordinary ($n=1$ dimensional) Virasoro factorization algebra and the Virasoro vertex algebra.
In \cite{CG1}, it is shown that the cohomology (evaluated on a formal disk) of appropriately structured \textit{holomorphic} factorization algebras on $\Sigma=\C$ are endowed with the structure of a vertex algebra.
This turns into a functor $\VV$ from a certain category of `tame' holomorphic factorization algebras on $\Sigma = \C$ to the category of $\Z$-graded vertex algebras.

\begin{thm}[\cite{BWvir}]
For $c \in \C$ consider the class
\begin{equation}
\frac{c}{12} c_1^2 \in H^4(BU(1))
\end{equation}
and the associated Virasoro factorization algebra $\Vir_{c c_1^2/12}$ on $\C$.
This is a tame holomorphic factorization algebra and its associated vertex algebra $\VV(\Vir_{c c_1^2/12})$ is equivalent to the Virasoro vertex algebra of central charge $c$.
\end{thm}

We now arrive at the main definition of this paper.

\begin{dfn}\label{dfn:hol}
A \defterm{holomorphic structure} on a factorization algebra $\cF$ on $\C^n$ of central charge $\bc \in H^{2n+2}(BU(n))$ is a map of factorization algebras
\begin{equation}
\bT \colon \Vir_{\bc} \to \cF .
\end{equation}
\end{dfn}

Largely, the goal of the rest of the paper is to produce examples of holomorphic structures on factorization algebras which are associated to ``free'' quantum field theories which will be introduced in section \ref{s:free}.
We will also study algebraic consequences of this structure in the form of representations for the higher Virasoro algebra.
We turn to this now.

\subsection{Relationship to higher Virasoro Lie algebras}

In this section we show how to produce central extensions of $\lie{witt}(n)$ using factorization algebras.
Let $\bc$ be as in definition \ref{dfn:vir} and consider the associated Virasoro factorization algebra on $\C^n$ by $\Vir_{\sfc}$.

\begin{lem}
There is an $U(n)$-action on the factorization algebra $\Vir_{\sfc}$ which covers the linear action on $\C^n$.
\end{lem}
\begin{proof}
Any holomorphic diffeomorphism induces an automorphism of the Dolbeault complex with coefficients in any natural holomorphic vector bundle.
In particular, there is a $GL(n,\C)$ action on the untwisted factorization enveloping algebra $\Vir_{0}$.

For the case $\bc \ne 0$ we need to see that the corresponding local functional is invariant under such rotations.
This is by construction: the local functional corresponding to $\mathbf{c}$ originates from an $n$-dimensional universal characteristic class, which is manifestly $U(n)$-invariant.
%If $\sum_i g_i(z) \del_{z_i}$ is a vector field then rotation by $\lambda \in S^1$, the $i$th circle, gives the vector field $\sum_i \lambda_i^{-1} g_i(\lambda_i z) \del_{z_i}$ \brian{A little stuck}
\end{proof}

We restrict the factorization algebra $\Vir_{\bc} |_{\C^n - \{0\}}$ to define a factorization algebra the punctured plane $\C^n - \{0\}$.
Let 
\begin{equation}
\rho \colon \C^n - \{0\} \to \R_+
\end{equation}
be the square of the radius $(z_1,\ldots,z_n) \mapsto |z_1|^2 + \cdots + |z_n|^2$, and consider the resulting factorization algebra 
\begin{equation}
\cF_{\bc} \define \rho_* \left(\Vir_{\bc} |_{\C^n - \{0\}}\right)
\end{equation}
on $\R_+$ whose value on an open interval $I \subset \R_+$ is the value of $\Vir_{\bc}$ on the open annular region $\rho^{-1}(I) \subset \C^n$.

Let $\cA_{0}$ be one-dimensional enveloping factorization algebra associated to the local Lie algebra $\Omega^{\bu}_{\R} \otimes \lie{witt}(n)$.
Its underlying associative algebra is the ordinary (or $\mathbb{E}_1$) enveloping algebra $U \lie{witt}(n)$.
The map $\til j$ from equation \eqref{eqn:tilj} defines an injective map of factorization algebras
\begin{equation}\label{eqn:injective}
\til j \colon \cA_{0} \to \cF_{0} .
\end{equation}


For $k \in \Z$, define the factorization algebra
\begin{equation}
\cA_{\bc} \define \bigoplus_{\bk \in \Z^m} \cF_{\bc}^{(k)} ,
\end{equation}
where $\cF_{\bc}^{(k)}$ denotes the weight $k$ eiegenspace for the diagonal $S^{1} \subset U(n)$ action.

\begin{prop}
The factorization algebra $\cF_{\bc}$ is a locally constant factorization algebra on~$\R_+$. 
Its associated $\mathbb{E}_1$-algebra is the $\mathbb{E}_1$-enveloping algebra of the dg Lie algebra $\vir_{\bc}$ given by the central extension
\begin{equation}
0 \to \C \to \lie{vir}_{\bc} (n) \to \lie{witt}(n) \to 0
\end{equation}
determined by the degree two cocycle which is the image of $\bc$ 
\begin{equation}
H^1_{loc}(\cT
\end{equation}
\end{prop}

\subsubsection{No central extension}

\subsubsection{Turning on the central extension}


%\section{Deligne truncations of de Rham cohomology}

\section{Free-field factorization algebras}
\label{s:free}

In this section we introduce the sort of `free' factorization algebras which we will use to realize symmetries by the higher dimensional Virasoro algebras introduced in the last section.
The contents of this section are only a mild generalization of \cite[\S 8.1]{CG2} which associates a factorization algebra of quantum observables to free quantum field theories.
In \textit{loc. cit.} the definition of a free quantum field theory requires non-degeneracy of the two-form on the space of fields; in other words, it pertains to the Batalin--Vilkovisky (BV) quantization of a $(-1)$-shifted symplectic space. 
In this section we weaken non-degeneracy defining the free classical field theory; or equivalently, we start with a type of $(-1)$-shifted presymplectic structure.
Because of the strict assumptions we place on the $(-1)$-shifted presymplectic structure, we can associate to it a $\PP_0$-factorization algebra.
We then form the BV quantization of the associated $\PP_0$-factorization algebra of classical observables.

In fact, there is a more direct approach to constructing this factorization algebra, which follows very closely the well-known Heisenberg algebra construction

\subsection{Weyl factorization algebras}
\label{ss:weyl}

Suppose that $V$ is a symplectic vector space with symplectic form $\eta$.
The Heisenberg Lie algebra is the central extension of $V$, viewed as an abelian Lie algebra, defined by $\eta$, viewed as a $2$-cocycle.
The non-degeneracy of~$\eta$ is not essential for this construction, any $\eta \in \wedge^2 V^*$ defines the central extension
%In fact, if $(V,Q)$ is a cochain complex, then any $Q$-closed element
%\begin{equation}
%\eta \in \Sym(V^*[-1])
%\end{equation}
%of total degree two defines a central extension of dg Lie algebras\footnote{In fact, if we just require that $\eta$ be of total cohomological degree $2$ in this graded symmetric algebra, then when we will find is $L_{\infty}$-central extensions of $V$.}
\begin{equation}
0 \to \C \to \lie{heis}(V,\eta) \to V \to 0 .
\end{equation}
The (dg) Weyl algebra associated to $V,\eta$ is the universal enveloping algebra of this central extension where the central parameter is set to the unit.
In this section we review a factorization algebra enhancement of this construction.

We will modify/generalize this construction in two ways:
\begin{itemize}
  \item[(1)] $V$ is a cochain complex and is equipped with a closed two-form $\eta$ of total degree $+1$.
  \item[(2)] $V$ is a complex of vector bundles over a (spacetime) manifold $M$. Then, we require this degree $+1$ two-form $\eta$ is ``local'' in a certain sense.
\end{itemize}

For the first item, we proceed totally analogously as in the usual Heisenberg algebra.
Suppose $(V,Q)$ is a cochain complex and $\eta$ is a $Q$-closed graded skew symmetric bilinear form on $V$ of total degree $+1$.
Then the shifted Heisenberg algebra is a dg Lie algebra central extension
\begin{equation}\label{eqn:shifted}
0 \to \C[-1] \to \lie{heis}(V,\eta) \to V \to 0 .
\end{equation}

Consider now a complex of vector bundles $(V,Q)$ on a manifold $M$.
So, now $V \to M$ is a $\Z$-graded vector bundle with sheaf of sections $\cV = \Gamma(M,V)$ and $Q \colon \cV \to \cV[1]$ is a square-zero differential operator of degree $+1$.
By applying the construction of the enveloping factorization algebra, recalled in \ref{s:enveloping}, to the abelian local Lie algebra $L = V[-1]$ we obtain the factorization algebra
\begin{equation}
\Sym(\cV_c) \colon U \subset M \mapsto \Sym(\cV_c(U)) .
\end{equation}
The differential is $Q$ which acts as a graded derivation.

In the context of vector bundles, we need to assume that the functional $\eta$ is local in the sense that it is given by the integral of a Lagrangian density.
Thus, we fix a local functional
\begin{equation}
\eta \in \oloc(V)[1]
\end{equation}
of cohomological degree $+1$ which satisfies $Q \omega = 0$.
In other words, this is a local cocycle for the abelian local Lie algebra $L = V[-1]$ of degree $+1$.
Notice that in this form, $\eta$ is not necessarily quadratic and could have arbitrary polynomial degree.
Although, for most examples $\eta$ will be quadratic (but not necessarily non-degenerate).

Associated to $\eta$ an operator $\triangle_{\eta}$ acting on $\Sym(\cV_{c})$ is defined.
It is an example of a Batalin--Vilkovisky (BV) operator, which we can describe explicitly.
The local functional $\eta$ defines, for each open set $U$, a $(-1)$-shifted central extension of $L_{\infty}$ algebras
\begin{equation}
0 \to \C k [-1] \to \cH\lie{eis}_c(U) \to \cV_c(U) \to 0 .
\end{equation}
where the nontrivial operations are the differential $Q$ and the higher $\ell$-ary operation
\begin{equation}
[f_1,\ldots,f_\ell]_\ell = \int_U \eta_k(f_1,\ldots,f_\ell) k
\end{equation}
where $\eta_{\ell}$ is the $\ell$th Taylor component of $\eta$ and $f_{i}\in \cV_c(U)[-1]$.
Notice that as graded vector spaces
\begin{equation}
\Sym\left(\cH\lie{eis}_c (U) [1] \right) = \Sym(\cV_c(U))[K]
\end{equation}
where $K$ is a parameter of degree zero.
The differential $Q + K \triangle_\eta$ acting is the Chevalley--Eilenberg differential for $\cH\lie{eis}_c(U)$, so
In other words, $\triangle_{\eta}$ is the component of the Chevalley--Eilenberg differential for $\cH\lie{eis}_{c}$ which does not involve $Q$.

This construction is a special case of the twisted enveloping factorization algebra construction.

\begin{prop}[\cite{CG1}]
The assignment
\begin{equation}
U \subset M \mapsto \bigg(\Sym(\cV_c(U))[K] \,, \, \,Q + K \triangle_\eta \bigg)
\end{equation}
has the structure of a factorization algebra in the category of dg $\C[K]$-modules, where $K$ is a variable of degree zero.
\end{prop}

Let $\cW_{V,\eta}[K]$ be the factorization algebra on $M$ valued in $\C[K]$-modules described in this proposition.

\begin{dfn}\label{dfn:weyl}
The \defterm{Weyl factorization algebra} associated to the complex of vector bundles $(V,Q)$ and the local functional $\eta \in \oloc(V)[1]$ of degree one is the $K=1$ specialization $\cW_{V,\eta} \define \cW_{V,\eta}[K]|_{K=1}$.
\end{dfn}

\begin{eg}
Let $V_0$ be an ordinary vector space and $\omega_{V_0} \in \wedge^2 V_0^*$ a symplectic form.
Consider the complex $\cV$ of vector bundles on the real line
\begin{equation}
\Omega^0(\R) [1]\otimes V_0 \xto{\d \otimes \id} \Omega^1 (\R) \otimes V_0 ,
\end{equation}
where $\Omega^0\otimes V$ is placed in degree $-1$.
Then $\omega = \int_\R \omega_{V_0}$ defines a local functional of degree $+1$ on this complex which satisfies~$\d \omega = 0$.
The Weyl factorization algebra in this case is locally constant on $\R$ and its associated $\EE_1$-algebra is equivalent to the ordinary Weyl associative algebra on the symplectic vector space~$V_0$.
More generally, if $V$ is an $n$-shifted symplectic vector space then the corresponding local functional on the complex of vector bundles $\Omega^\bu(\R^n) \otimes V$ is also degree $+1$.
The associated Weyl factorization algebra is locally constant and is a $\mathbb{E}_n$-algebra analog of the ordinary Weyl algebra.
\end{eg}

\begin{eg}
  Let $\Sigma$ be a Riemann surface and consider the complex $\cV$ of vector bundles
  \begin{equation}
    \Omega^{0,\bu}(\Sigma)[1] = \Omega^{0,0}(\Sigma)[1] \xto{\dbar} \Omega^{0,1}(\Sigma)
  \end{equation}
  concentrated in degree $-1$ and zero.
  A section of $\cV$ is denoted $\alpha$.
  Define $\eta$ to be the local functional
  \begin{equation}
    \eta = \frac12 \int_{\Sigma}\alpha \del \alpha .
  \end{equation}
  Clearly $\dbar \eta = 0$.
  The Weyl factorization algebra associated to $\cV,\eta$ is the factorization algebra underlying the free chiral boson CFT \cite[??]{CG}.
\end{eg}

Recall that the sheaf of local functionals associated to $V$ can be expressed in terms of polydifferential operators as
\begin{equation}
\oloc(V)= \prod_{n \geq 0} \op{PolyDiff}(\cV^{\times n} , \op{Dens}_{M})_{{S_{n}}}
\end{equation}
where $\op{Dens}_{M}$ is the bundle of densities on $M$.
In each of the above examples the functional $\eta$ is quadratic.
For simplicity, we will assume this from hereon.
Thus, we can equally view $\eta$ as a graded symmetric section
\begin{equation}
\eta \in \Gamma(M, \op{Diff}(\cV, \cV^!))[1]
\end{equation}
where $\cV^{!}$ is the sheaf of sections of the Serre dual vector bundle $V^{!} = V^{*} \otimes \op{Dens}_M$.

%Assume that $\eta$ is at least quadratic and consider the factorization algebra $\cW_{V,\eta}$.
For each open set $U \subset M$, there is an increasing filtration on $\Sym(\cV_c(U))$ by polynomial degree. 
This induces a filtration on $\cW_{V,\eta}$ as a factorization algebra.
\begin{prop}
The associated graded factorization algebra $\op{gr} \cW_{V,\eta}$ assigns to an open set $U \subset M$ the cochain complex
\begin{equation}
 \bigg(\Sym(\cV_c(U)) \,, \, \,Q \bigg) .
\end{equation}
Notice that this factorization algebra is independent of $\eta$.
\end{prop}

Notice that this associated graded factorization algebra agrees with the $K=0$ specialization of $\cW_{V,\eta}[K]$.
The component of the Chevalley--Eilenberg differential $Q_\eta$, which is missing from $\op{gr} \cW_{V,\eta}$,  induces a structure on the associated graded factorization algebra.

\textbf{Note}: From here on we will assume that $\eta$ is purely a quadratic local functional.

Unlike $Q$--the differential induced from the linear differential acting on sections of the complex of vector bundles $V$--the operator $\triangle_\eta$ is not a derivation.
It induces a bracket on the associated graded factorization algebra.
To motivate it, we step aside to explain the following general construction in the setting of Lie algebras.

Given an arbitrary dg Lie algebra $(\lie{g},\d_g)$, its Chevalley--Eilenberg complex is $C_\bu(\lie{g}) = \left(\Sym(g[1]), \d_{CE}\right)$ where $\d_{CE}$ is the total Chevalley--Eilenberg differential which is a sum of the linear differential $\d_{\lie{g}}$ plus the Chevalley--Eilenberg differential $\d_{[-,-]}$ associated to the bracket.
Consider the filtration on $C_\bu(\lie{g})$ by polynomial degree. 
The associated graded of cochain complex is $\op{gr} C_\bu(\lie{g}) = (\Sym(\lie{g}[1]) , \d_{\lie{g}})$; so, like the Chevalley--Eilenberg complex except the differential is just the linear part.
Since $\d_{\lie{g}}$ is a derivation, we see that $\op{gr} C_\bu(\lie{g})$ is a commutative dg algebra.
For $F,G \in \op{gr} C_\bu(\lie{g})$ we define
\begin{equation}\label{eqn:P0}
\{F,G\} = \d_{[-,-]} (FG) - \d_{[-,-]} (F) G \pm F \d_{[-,-]} (G) .
\end{equation}
For example, if $F = x, G = y \in \lie{g}[1] \subset \op{gr} C_\bu(\lie{g})$ then $\{x,y\} = [x,y]$, the original Lie bracket.
The formula \eqref{eqn:P0} defines a bracket on $\op{gr} C_\bu(\lie{g})$ of cohomological degree $+1$, satisfies the graded Jacobi identity, is a graded derivation for the commutative product on,  and satisfies the Leibniz rule with respect to $\d_{\lie{g}}$.
Thus, it endows $\op{gr} C_\bu(\lie{g})$ with the structure of a $\PP_0$-algebra.

We are in a special case of this situation.
For $\Phi,\Psi$ in $\op{gr} \cW_{V,\eta} (U)$ we define
\begin{equation}
\{F, G\} \define Q_{\eta} (F G) - Q_{\eta}(F) G \pm F Q_{\eta}(G) . \footnote{Notice that we could replace $Q_{\eta}$ by the full differential $Q + Q_{\eta}$ in this formula.}
\end{equation}
This bracket is of cohomological degree $+1$, satisfies the graded Jacobi identity, is a graded derivation for the commutative product on $\op{gr} \cW_{V,\eta} (U)$, and satisfies the Leibniz rule with respect to $Q$.
In other words, it endows $\op{gr} \cW_{V,\eta}$ with the structure of a $\PP_0$-factorization algebra \cite[??]{CG2}.

\subsection{Local functionals of a free theory}\label{s:localfree}

Let $(E,Q)$ be a complex of vector bundles and consider its sheaf of local functionals $\oloc(E)$.
The complex of local one-forms $\Omega^1_{loc}(E)$ is defined similarly.

Concretely
\begin{equation}
\Omega^1_{loc}(E) \subset \prod_{n \geq 0} \Hom_{\C} (E^{\otimes n}, E^!)
\end{equation}
is the subcomplex of multilinear operators which are differential operators.
The exterior derivative defines a map
\begin{equation}
\d \colon \oloc(V^!) \to \Omega^1_{loc} (V^!) .
\end{equation}

Suppose now that $E = V^!$ and $\eta \in \oloc(V)$ is a quadratic local functional of degree~$+1$.
Then $\Omega^1_{loc}(E) = \Omega^1_{loc}(V^!)$. 
The following lemma is a special case of \cite[theorem ??]{ButsonYoo} but we provide a self-contained proof as it contains formulas that we will use in the proceeding sections.

\begin{lem}
The functional $\eta$ defines a
%\begin{equation}
%\<-,-\>_\eta \colon \Omega^1_{loc} (V^!) \times \Omega^1_{loc}(V^!) \to \oloc(V) [1]  .
%\end{equation}
graded Lie bracket
\begin{equation}
\{-,-\} \colon \oloc(V^!) \times \oloc(V^!) \to \oloc(V^!)[1]
\end{equation}
of cohomological degree $+1$.
\end{lem}
\begin{proof}
Suppose that the local functional $\Phi \in \oloc(V^!)$ is of polynomial degree $k$, meaning we can write it as a finite sum
\begin{equation}
\Phi = \sum_i \omega D^\Phi_{i,1} \cdots D^\Phi_{i,k}
\end{equation}
where $\omega$ is a section of $\op{Dens}_M$ and $D_{i,j}$ are differential operators $\cV^! \to C^\infty_M$.
This local functional defines the linear map
\begin{equation}
\int \Phi \colon \cV_c^!(M)^{\otimes k} \to \C
\end{equation}
defined by
\begin{equation}
\alpha_1 \otimes \cdots \otimes \alpha_k \mapsto \sum_i \int_M \omega  (D_{i,1}^\Phi \alpha_1) \cdots (D_{i,k}^\Phi \alpha_k) .
\end{equation}

The differential $\d \Phi$ is of the form
\begin{align*}
\d \Phi & \colon \cV^!_c(M)^{\otimes (k-1)} \to \cV_c(M)  \\ & \alpha_1 \otimes \cdots \otimes \alpha_{k-1} \mapsto \sum_i \omega (D_{i,1}^\Phi \alpha_1) \cdots (D_{i,k-1}^\Phi \alpha_{k-1}) D_{i,k}(-) + \cdots 
\end{align*}
where the $\cdots$ denotes graded symmetrization.
The differential $\d \Phi (\alpha_1 \otimes \cdots \otimes \alpha_{k-1})$ is \textit{a priori} a distributional section of $\cV_c$, but upon integrating by parts we see that it is, in fact, a smooth section.

Now, suppose $\Psi$ is of polynomial degree $\ell$.
We have the composition
\begin{equation}
\cV_{c}^!(M)^{\otimes (k+\ell-2)} \xto{\d \Phi \otimes \d \Psi} \cV_c(M)^{\otimes 2} \xto{\eta} \Gamma_c(M,\op{Dens}_M)
\end{equation}
that we denote by $\eta\<\d \Phi, \d \Psi\>$.
%Since polydifferential operators preserve the condition of being compactly supported, we see that $\eta\<\d \Phi, \d \Psi\>$ is compactly supported.
Define
\begin{equation}
\int_M \{\Phi, \Psi\} \define \int_M \eta\<\d \Phi, \d \Psi\> \colon \cV_c^!(M)^{\otimes (k+\ell-2)} \to \C .
\end{equation}
This is clearly a polydifferential operator, so we see that this construction defines a local functional $\{\Phi,\Psi\} \in \oloc(V^!)$.
\end{proof}





%We consider the Serre dual $V^!$ as a complex of vector bundles.
%As we've utilized, integration over $U \subset M$ defines an inclusion of cochain complexes
%\begin{equation}
%\oloc(V^!) \hookrightarrow \Sym(\cV_c(U)) .
%\end{equation}
%We will denote the image of a local functional $\Phi$ under this embedding by $\int_U \Phi$.
%
%\begin{lem}
%Suppose that $\Phi, \Psi$ are local functionals on $V^!$, which we may view as elements in $\Sym(\cV_c(U))$ for any open set $U \subset M$.
%Then there exists a unique local functional $\Xi$ such that $\int_U \Xi = \{\int_U \Phi, \int_U \Psi\}$.
%\end{lem}

This lemma defines a bracket on local functionals which is of cohomological degree~$+1$.
Together with the differential $Q$ this bracket thus endows $\oloc(V^!)[-1]$ with the structure of a dg Lie algebra.

\subsection{Quantization in the BV formalism}

The Weyl algebra is the result of deformation quantization of the simplest symplectic space.
In this section we show that the Weyl factorization algebra is also the result of a sort of quantization in the context of the Batalin--Vilkovisky (BV) formalism.

The following is a special case of the construction in the previous subsection.
Suppose $(\cE,Q, \omega)$ is a free BV theory in the sense of \cite[definition 4.2.0.2]{CG2}.
Here, $(\cE,\omega)$ is an elliptic complex of vector bundles and
\begin{equation}\label{eqn:nondeg}
\omega \colon E \otimes E \to \op{Dens}_M [-1]
\end{equation}
is a bundle map that is fiberwise non-degenerate.

Recall that the $!$-dual (or Serre dual) of a vector bundle $E$ is the bundle $E^! = E^* \otimes \op{Dens}_M$ where $E^*$ is the ordinary dual vector bundle and $\op{Dens}_M$ is the density bundle on $M$ (an orientation is an isomorphism $\op{Dens}_M \cong \wedge^n \T_M^*$).

The section $\omega$ defines a graded bundle isomorphism $E^! \simeq E[1]$.
In particular, there is a sequence of graded bundle isomorphisms
\begin{align*}
E^* \otimes E^* \otimes \op{Dens}_M [-1] & \simeq E^! \otimes E^! \otimes \op{Dens}_M^{-1} [-1] \\ & \simeq_{\omega} E \otimes E \otimes \op{Dens}_M^{-1} [-1] \\ & \simeq (E^!)^* \otimes (E^!)^* \otimes \op{Dens}_M [1]
\end{align*}
We gather that at the level of compactly supported sections there is an embedding
\begin{equation}
i \colon \Gamma_c (M, E^* \otimes E^* \otimes \op{Dens}_M) \hookrightarrow \oloc(E^!) .
\end{equation}
The image of $\omega$ is thus a $Q$-closed local functional $i(\omega) \in \oloc(E^!)$ for the vector bundle~$E^!$.

So, if we set $V = E^!$ and $\eta = i(\omega)$, the construction of the Weyl factorization algebra of the previous section yields the factorization algebra $\cW_{E^!, i(\omega)}$.
This is the factorization algebra of observables of the free BV theory $(E,Q,\omega)$ as in \cite{CG2}.

It is in this sense that $\cW_{V,\eta}$ is a generalization of ordinary BV quantization of the BV theory whose ``fields'' are sections of the graded vector bundle $E = V^!$.
Generally, there is no non-degeneracy requirement on the local functional $\eta \in \oloc(V)$.

%First, we recall some notation.
%
%The linear dual of $\cE = \Gamma(M,E)$ is $\br \cE^!_c = \br \Gamma_c(M, E^!)$ where $\br \Gamma_c$ denotes compactly supported distributional sections.
%There is hence a natural map of cosheaves
%\begin{equation}
%\cE^!_c = \Gamma_c(M,E^!) \hookrightarrow \cE^\vee ,
%\end{equation}
%which includes the smooth (non-distributional) sections.
%If $E$ is an elliptic complex of vector bundles then the above map is a quasi-isomorphism.
%In particular, one has the following.
%
%\begin{prop}[see theorem 5.4.0.1 of \cite{CG}]
%Let $(E,Q)$ be an elliptic complex of vector bundles.
%Then the inclusion
%\begin{equation}
%\left(\Sym(\cE_c), Q\right) \hookrightarrow \left(\cO(\cE^!) , Q\right)
%\end{equation}
%is an equivalence of factorization algebras.
%If $\omega \in \oloc(E)$ is non-degenerate of the form \eqref{eqn:nondeg} then this is an equivalence of $\PP_0$-factorization algebras.
%\end{prop}
%
%The following paragraph is a review of ideas that appear in \cite{PavelPoisson}.
%Let $(P,\d)$ be a cochain complex. Consider the following commutative dg algebra
%\begin{equation}
%\op{Pol}(P,-1) \define \Sym(P^*) \otimes \Sym(P) .
%\end{equation}
%We consider two additional gradings on this algebra.
%The first is by polynomial degree where $P^*$ carries weight $+1$.
%The second endows $P$ with weight $+1$, we refer to this as the weight grading.
%The (shifted) Schouten bracket $[-,-]_{Sch}$ endows this with the structure of a graded (unshifted) Lie algebra. 
%It is defined on this usual way on vector fields $\Sym(P^*) \otimes P \subset \op{Pol}(P,-1)$ and extended to the full algebra by the condition that it is a graded derivation with respect to multiplication.
%A $(-1)$-shifted Poisson structure on $P$ is an element
%\begin{equation}
%\pi \in \op{Pol}(P,-1) [1]
%\end{equation}
%of cohomological degree $+1$, concentrated in weights $\geq 2$, which satisfies the Maurer--Cartan equation
%\begin{equation}
%\d \pi + \frac12 [\pi,\pi]_{Sch} = 0 .
%\end{equation}
%Such an element can be decomposed $\pi = \pi_2 + \pi_3 + \cdots$ where $\pi_k$ has weight $k$ with respect to the weight grading.
%Contraction with $\pi$ endows the commutative dg algebra $\cO(P)$ with a sequence of brackets $\{-,-\}, \{-,-\}_3, \ldots$ endowing it with the structure of a \textit{homotopy $\PP_0$ algebra} \cite{??}.
%In this paper we only consider $(-1)$-shifted Poisson structures with weight $2$, hence $\cO(P)$ is an ordinary, or strict, $\PP_0$-algebra with bracket we denote by $\{-,-\}$.
%In physics, this is the BV anti-bracket.
%
%Given such a $\pi$ we construct an operator $\triangle_\pi$ acting on $\cO(P)$ defined as follows.
%Let 
%\begin{equation}
%\triangle_{\pi_k} \colon \Sym^k(P^*) \to \C 
%\end{equation}
%be contraction with $\pi_k$; extend this to ... \brian{finish}
%
%
%
%\begin{prop}
%Let $(P,\d)$ be a cochain complex and suppose that $\pi$ is a constant coefficient $(-1)$-shifted Poisson structure on $P$.
%Then $\triangle_\pi \circ \triangle_\pi = 0$.
%\brian{BD quantization??}
%\end{prop}
%
%\begin{dfn}\label{dfn:free}
%A \defterm{free BV theory of Poisson type} on a manifold $M$ is an elliptic complex of vector bundles $(\cE,Q)$ on $M$ together with a local, constant coefficient, $(-1)$-shifted local Poisson tensor
%\begin{equation}
%\pi \in ??
%\end{equation}
%which satisfies $Q \pi = 0$.
%\end{dfn}
%
%Now we return to the relationship to Weyl factorization algebras.
%A $\PP_0$-structure $\pi$ as in definition \eqref{dfn:free} 
%The factorization algebra of classical observables of a free BV theory of Poisson type is simply $\cO(\cE) = \Sym\left(\cE^\vee\right)$.
%

\section{Examples of holomorphic free field factorization algebras}
\subsection{The higher $\beta\gamma$ system}

If $X$ is a complex manifold, and $p$ a non-negative integer, let
\begin{equation}
\Omega^{p,\bu}(X) = \Gamma_{C^\infty}\left(X, \Sym(\br \T^*_X [-1]) \otimes \wedge^p \T^*_X\right)
\end{equation}
be the space of smooth sections of the graded vector bundle of $(p,\bu)$ forms on $X$, where $(p,q)$ forms sit in cohomological degree $q$.
This complex of bundles is equipped with the natural $\dbar$ operator turning $\Omega^{p,\bu}(X)$ into a dg $\Omega^{0,\bu}(X)$-module.
Similarly, $\Omega_{c}^{p,\bu}(X)$ is the subcomplex of compactly supported forms on $X$.
The complex of sheaves $U \mapsto \Omega^{p,\bu}(U)$ is a free resolution for the sheaf of holomorphic $p$-forms $\Omega^{p,hol}_{X}$.

We move on to our first example of this section.
Suppose that $X$ is a complex manifold of dimension $n$.
Define $\cV = \Gamma(X,V)$ to be the complex of vector bundles
\begin{equation}\label{eqn:bg}
\Omega^{0,\bu}_{X} [1] \oplus \Omega^{n,\bu}_{X} [n],
\end{equation}
whose sections we will denote by $(\beta,\gamma)$.
The local functional $\eta$ is defined by
\begin{equation}\label{eqn:etabg}
\eta = \int \beta \wedge \gamma .
\end{equation}

The $\beta-\gamma$ \defterm{factorization algebra} on $X$ is the Weyl factorization algebra associated to this data, see definition \ref{dfn:weyl}.
We denote it by $\beta\gamma \define \cW_{V,\eta}$.

More generally, given any super vector bundle $Z \to X$, one can consider the $\beta-\gamma$ system valued (or twisted by) $Z$.
In this case $\cV$ is the complex of bundles
\begin{equation}\label{eqn:bgZ}
\Omega^{0,\bu}_{X}(Z) [1] \oplus \Omega^{n,\bu}_{X}(Z^{*}) [n].
\end{equation}
where $Z^{*}$ is the dual bundle.

In super string theory, the plain $\beta-\gamma$ system on a Riemann surface $X = \Sigma$ plays a role in BRST quantization.
In fact, the cohomology of the factorization algebra $\beta\gamma$ on~$\Sigma=\C$ recovers the usual $\beta-\gamma$ vertex algebra \cite{CG1}.

When $n=2$, the $\beta-\gamma$ system is the factorization algebra of observables of the holomorphic twist of the $\cN=1$ supersymmetric quantum field theory of a free chiral multiplet in four (real) dimensions \cite{CosYangian,ESW}.
When $n=3$ and $Z = \Pi K^{1/2}_{X}[1]$ (this is a choice of a spin structure on the threefold $X$), the $\beta-\gamma$ system is equivalent to the holomorphic twist of the $\cN=(1,0)$ supersymmetric theory of a free hypermultiplet in six (real) dimensions \cite{ESW,SWtensor}.

There are relationships among $\beta-\gamma$ systems of different dimensions.
For example, the plain $(n+m)$-dimensional $\beta-\gamma$ system can be defined on the manifold $X \times \PP^{m}$.
Its compactification along $\pi \colon X \times \PP^{m} \to X$ is simply the $n$-dimensional $\beta-\gamma$ system on $X$.
That is, there is an equivalence of factorization algebras
\begin{equation}\label{eqn:bgcompact}
\pi_{*} \beta\gamma[X \times \PP^{m}] \simeq \beta \gamma[X] .
\end{equation}
Here, $\beta\gamma[Y]$ denotes the $\beta-\gamma$ factorization algebra on the complex manifold $Y$.

\subsection{The higher Kac--Moody factorization algebra for $\lie{gl}(1)$}

Let $X$ be a complex manifold of dimension $n$.
Consider the shifted Dolbealt complex
\begin{equation}\label{eqn:gl1}
\cV = \Omega^{0,\bu}(X)[1] .
\end{equation}
On this complex of bundles, there is the following $(n+1)$-linear local functional
\begin{equation}\label{eqn:km}
\eta = \frac{1}{(n+1)!} \int \alpha \del \alpha \cdots \del \alpha .
\end{equation}
Though it is not quadratic (like our previous examples), it is nevertheless of total cohomological degree $+1$ and defines a factorization algebra via the construction of definition \ref{dfn:weyl}.
We denote it by $\cU = \cW_{V,\eta}$ and refer to it as the (higher-dimensional) \defterm{$\lie{gl}(1)$ Kac-Moody factorization algebra}.
(Notice that in this example we do not claim to construct a bracket on local functionals on $E = V^{!}$ as in the setting of section \ref{s:localfree}.)

More generally, $\cU$ is an example of a higher-dimensional Kac--Moody factorization algebra as defined in \cite{GWkm}.
For any Lie algebra $\lie{g}$, we can define the enveloping factorization algebra (see section \ref{s:enveloping}) associated to the local Lie algebra $\lie{g} \otimes \Omega^{0,\bu}(X)$.
(This is not an example of a ``Weyl factorization algebra'' unless $\lie{g}$ is itself abelian.)
Moreover, this can be centrally extended by a cocycle of the form $\eta$ which is defined, more generally, with the additional data of an invariant polynomial for $\lie{g}$ of degree $n+1$.
In the case $\lie{g} = \lie{gl}(1)$ then this is no extra data at all and we recover the factorization algebra $\cU$.
More generally, if $\theta$ is such a degree $(n+1)$ polynomial on a Lie algebra $\lie{g}$, the cocycle is
\begin{equation}\label{eqn:km2}
\eta_\theta = \frac{1}{(n+1)!} \int \theta(\alpha \del \alpha \cdots \del \alpha) .
\end{equation}
This cocycle first appeared in the work of \cite{FHK} where they used it to construct central extensions of the higher dimensional current algebra $\R\Gamma(\mathring{D}^{n},\cO) \otimes \lie{g}$.

%In \cite{GWkm} we find a relationship between $\cU$ and the $\beta\gamma$, introduced in the previous section.

%It is straightforward to observe the following relationship among the $\lie{gl}(1)$ Kac--Moody factorization algebras of different dimension.
%For instance, if $\pi \colon X \times \P^{m} \to X$ is the pushforward, then
%\begin{equation}\label{eqn:kmcompact}
%\pi_{*}\cU[X \times \PP^{m}] \simeq \cU[X] .
%\end{equation}



\subsection{The higher-dimensional free chiral boson}

The previous two examples exist in any complex dimension.
This class of examples only exists in odd complex dimensions.
Let $X$ be a $(2n+1)$-dimensional complex manifold.
Consider the holomorphic de Rham operator
\begin{equation}
\del \colon \Omega^{p,\bu}_X \to \Omega^{p+1,\bu}_X
\end{equation}
acting on Dolbeault forms.
The total complex of sheaves
\begin{equation}
\left(\Omega^{\bu,\bu}_X , \dbar + \del\right) = \bigg(\oplus_{p,q} \Gamma_{C^\infty}\left(X, \wedge^q \br T^*_X \otimes \wedge^p \T^*_X\right) [-p-q] \; , \; \dbar + \del \bigg) 
\end{equation}
is the (complexified) smooth de Rham complex $\Omega^\bu_X$ equipped with the smooth de Rham differential $\d_{dR} = \dbar + \del$.

We are interested in a truncation of the full de Rham complex.
Consider the following complex of vector bundles
\begin{equation}
\Omega^{\geq p, \bu}_X \define \left(\Omega^{p,\bu} \xto{\del} \Omega^{p+1,\bu}[-1] \xto{\del} \cdots \right) 
\end{equation}
where the $\dbar$ operator is left implicit.
Explicitly, a section of this sheaf of cohomological degree $k$ is a sum of $(p',q)$ forms where $p' + q = k$ and $p' \geq p$.

For $X$ a smooth complex manifold, it is well-known that the complex of sheaves $\Omega^{\geq p, \bu}_X$ is a free resolution of the sheaf of $\del$-closed, holomorphic $p$-forms on $X$:
\begin{equation}
\Omega^{k,hol,cl}_X = \cK\op{er} (\del) \subset \Omega^{k,hol}_X .
\end{equation}

Define $(E,Q)$ to be the complex of vector bundles
\begin{equation}
\Omega^{\geq n+1, \bu}_X [n] .
\end{equation}
The differential $Q$ is the component the smooth de Rham operator $\dbar + \del$.
In other words, this is a subcomplex of the full de Rham complex.
This is the complex resolving the sheaf of closed and holomorphic $(n+1)$-forms placed in cohomological degree $-n$.
Thus, in this complex of bundles forms of type $(p,q)$ live in cohomological degree~$p+q-2n-1$.

We will denote the Serre dual by $V = E^!$, which inherits the structure of a complex of vector bundles.
Explicitly, the sheaf of sections of the Serre dual is
\begin{equation}\label{eqn:bosonV}
  \Omega^{\leq n, \bu}_X [2n+1] = \left(\Omega^{0,\bu} [2n+1] \xto{\del} \Omega^{1,\bu}[2n] \xto{\del} \cdots \Omega^{n,\bu} [n+1]\right) .
\end{equation}
So, in $V$, forms of type $(p,q)$ also live in cohomological degree $p+q-2n-1$ and the differential $Q$ is the components of the smooth de Rham operator $\dbar + \del$.
We use $\alpha$ to denote a section of $\cV$.

\begin{lem}
The local functional $\eta \in \oloc(V)$ defined by
\begin{equation}\label{eqn:bosoneta}
\eta = \frac12 \int \alpha \del \alpha 
\end{equation}
is of cohomological degree one and satisfies $Q \eta = 0$.
\end{lem}
\begin{proof}
$\dbar \eta = 0$ is obvious and $\del \eta = 0$ follows from $\del^2 = 0$.
\end{proof}

Define the (higher-dimensional) \defterm{chiral boson factorization algebra} $\cB$ on the $(2n+1)$-dimensional complex manifold $X$ to be the Weyl factorization algebra associated to this data, $\cB = \cW_{V,\eta}$.

Take $n=0$, so $X=\Sigma$ is a Riemann surface.
Then $\cE = \Omega^{1,\bu}_\Sigma$ and $\cV = \Omega^{0,\bu}_\Sigma [1]$.
The factorization algebra $\cB$ describes the (perturbative) chiral boson conformal field theory.
Indeed, in \cite[??]{CG1} it is shown that on $\Sigma = \C$ this factorization algebra produces the chiral boson vertex algebra consisting of a single field $\sfb$ of conformal weight one and OPE
\begin{equation}
\sfb(z) \sfb(w) \simeq \frac{1}{(z-w)^2} .
\end{equation}
In fact, we already covered this example.
It is identical to the $\lie{gl}(1)$ Kac--Moody factorization algebra on a Riemann surface.
This is a low-dimensional isomorphism, generally the chiral boson factorization algebra and (higher) $\lie{gl}(1)$
Kac--Moody factorization algebras are different.

There is the following relationship amoung the chiral boson factorization algebras of varying dimensions.
First, we can consider the chiral boson factorization algebra on $X \times \PP^{2n}$ where $X$ is a complex manifold of dimension $2n+1$.
Its pushforward, or compactification, along $\PP^{2n}$ returns the ordinary chiral boson factorization algebra on~$C$.

\begin{prop}
  Let $C$ be a Riemann surface.
  If $\pi \colon C \times \PP^{2n} \to C$ is projection, then there is an equivalence of factorization algebras
\begin{equation}\label{eqn:bosoncompact}
\pi_{*} \cB[C \times \PP^{2n}] \simeq \cB[C] \otimes \cT
\end{equation}
where $\cT$ is a locally constant factorization algebra on $X$.
\end{prop}
\begin{proof}
  The value of the factorization algebra $\pi_{*} \cB[X \times \PP^{2m}]$ on an open set $U \subset X$ is the graded symmetric algebra on
  \begin{equation}\label{}
\Omega_{c}^{\leq n+m, \bu}(U \times \PP^{m})[2n+2m+1] \simeq \Omega^{\leq n,\bu}_{c}(U) [2n+1] .
  \end{equation}
\end{proof}


%Let $X$ be a $(2n+1)$-dimensional complex manifold and consider the complex of vector bundle $\cA = \cA_X$ from definition \ref{dfn:??}.
%As a graded vector bundle recall that $\cA = \Omega^{\leq n, \bu} [n+1]$.
%
%Define the following pairing on the complex of vector bundles $\cA_{X,c}$:
%\begin{equation}
%\omega(\beta, \gamma) = \int_X \beta \del \gamma .
%\end{equation}
%One can immediately check that $\omega$ carries degree $+1$ as required for the construction of the Weyl factorization algebra.
%Importantly, for each open $U \subset X$ the pairing $\omega$ defines the operator $\triangle$ acting on 
%\begin{equation}
%\Sym \left(\cA_c(U) [n+1] \right)
%\end{equation}
%by the following rules:
%\begin{itemize}
%\item If $\Phi \in \Sym^{\leq 1}$ then $\triangle \Phi = 0$.
%\item If $\Phi = \beta \gamma \in \Sym^2$ then define
%\begin{equation}
%\triangle (\beta \gamma) = \omega|_{U} (\beta, \gamma) = \int_U \beta \del \gamma .
%\end{equation}
%\item \brian{FINISH}.
%\end{itemize}
%
%\begin{dfn}
%Let $X$ be a complex manifold of dimension $2n+1$.
%The \defterm{free boson factorization algebra} on $X$ is
%\begin{equation}
%\cF \define \left(\Sym \left(\Omega^{\leq n,\bu}_{X,c} [n+1] \right) \; , \; Q + \triangle \right) ,
%\end{equation}
%where $\triangle$ is defined as in items (1)-(3) above.
%In the notation of section \ref{ss:weyl}, $\cF = \cW_{\cA,\omega}$.
%\end{dfn}
%
%Finally, we frame this factorization algebra as a quantization of a free BV theory.
%First, let
%\begin{equation}
%\delta_\Delta \in \br\Omega^{2n+1,2n+1}(X \times X)
%\end{equation}
%be the $\delta$-distribution along the diagonal $X \hookrightarrow X \times X$.
%This is a distributional form of Hodge type $(2n+1,2n+1)$ on $X \times X$.
%Applying the holomorphic de Rham operator along one of the directions yields a distribution
%\begin{equation}\label{eqn:fullbcov}
%(\del \otimes \id) \delta_{\Delta} \in \br\Omega^{2n+2,2n+1}(X \times X) .
%\end{equation}
%This is the full BCOV propagator as introduced in \cite{CLbcov1,CLbcov2}.
%We are only interested in a particular component of this distributional section.
%There is a K\"unneth decomposition
%\begin{equation}
%\br\Omega^{2n+2,2n+1}(X \times X) = \bigoplus_{k=1}^{2n+1} C^k ,
%\end{equation}
%where $C^k = \oplus_{q+q'=2n+1} \br \Omega^{k,q}(X) \hotimes \br \Omega^{2n+2-k,q'}(X)$.
%We let
%\begin{equation}\label{eqn:poisson1}
%\pi \in C^{n+1} = \bigoplus_{q+q'=2n+1} \br \Omega^{n+1,q}(X) \hotimes \br \Omega^{n+1,q'}(X) 
%\end{equation}
%be the projection of \eqref{eqn:fullbcov} to the component $C^{n+1}$.
%
%\begin{lem}
%This $\pi$ determines a section
%\begin{equation}
%\pi \in \br \fields_X \hotimes \br \fields_X [-1]
%\end{equation}
%which satisfies $\d_{\fields} \pi = 0$.
%Thus $\pi$ endows $\fields$ with the structure of a free BV theory of Poisson type.
%\end{lem}
%
%\begin{prop}
%Let $X$ be a complex manifold of dimension $2n+1$.
%The free boson factorization algebra $\cF$ is the BV quantization of the free BV theory of Poisson type whose fields are
%\begin{equation}
%\fields = \Omega^{\geq n+1, \bu}_X [n]
%\end{equation}
%and whose $(-1)$-shifted local Poisson structure is $\pi$ as in \eqref{eqn:poisson1}.
%More precisely, there is an equivalence of factorization algebras
%\begin{equation}
%\cF \xto{\simeq} \Obs^q (\fields)|_{\hbar = 1} .
%\end{equation}
% \end{prop}

The chiral boson factorization algebras appear in string theory and supersymmetry.
Suppose that $X$ is a complex manifold of dimension three.
The chiral boson factorization algebra $\cB[X]$ is equivalent to the factorization algebra of observables in the holomorphic twist of the six-dimensional $\cN=(1,0)$ supersymmetric ``tensor multilpet'' theory \cite{SWtensor}.
Concretely, the field $\beta^{2,1} + \beta^{3,0} \in \cE^{0}$ represents the components of the field strength of the infamous chiral two-form \cite{WittenM5} which survive the holomorphic twist.

Suppose that $X$ is a complex manifold of dimension five.
The factorization algebra $\cB[X]$ represents the factorization algebra underlying the (field strength) of the Ramond--Ramond four-form in the holomorphic twist of type IIB supergravity.
Note that to understand this within supergravity, the manifold $X$ must be equipped with a Calabi--Yau structure \cite{CLsugra}.
Nevertheless, the factorization algebra $\cB[X]$ is defined independently of such a structure.

The following relationship links a few of the free-field holomorphic factorization algebras we have discussed.

\begin{prop}
  Let $X$ be a complex threefold and consider the chiral boson factorization algebra on $X \times \PP^{2}$.
  Then, there is an equivalence of factorization algebras
  \begin{equation}\label{eqn:20}
    \pi_{*} \cB[X \times \PP^{2}] \simeq \cB[X] \otimes \beta\gamma[X] \otimes \cT
  \end{equation}
  where $\cT$ is a locally constant factorization algebra on $X$ and $\pi \colon X \times \PP^{2}\to X$ is projection.
  Furthemore, this factorization algebra is equivalent to the observables underlying the holomorphic twist of the (abelian) $\cN=(2,0)$ superconformal theory on $X$.
\end{prop}

In general, we also have the following relationship to higher form Chern--Simons theory.

\begin{prop}[\cite{GRWcs}]
Let $X$ be a complex manifold of complex dimension $2n+1$. Consider $(2n+1)$-form Chern--Simons theory on the $(4n+3)$-dimensional product manifold with boundary $X \times \R_+$ equipped with the ``chiral'' boundary condition along $X \times \{0\}$ as defined in \cite{GRWcs}.
The boundary factorization algebra of quantum observables is equivalent to the chiral boson factorization algebra $\cB$ on $X$.
\end{prop}

\section{Segal--Sugawara construction}

In quantum field theory, it is often natural to study not a single theory but a family of theories over a derived stack $\cX$.
In the case that the derived stack is $\cX = BG$, with $G$ a group, this amounts to prescribing a $G$-symmetry on the quantum field theory.
As an example, consider the $\beta-\gamma$ system on a complex manifold $X$ of dimension $n$ as a family over the
derived stack $\cX = \op{Bun}_{GL(r)}(X)$ of vector bundles on $X$ of rank $r$.
The fiber over a vector bundle~$Z$ of rank $r$ is the $\beta-\gamma$ system twisted by $Z$, see equation \eqref{eqn:bgZ} for a description of the fields of this theory.

As a more elaborate example, also involving the $\beta-\gamma$ system, we consider $\cX$ to be the derived moduli stack of complex manifolds of dimension $n$.
Fixing a tensor bundle $\cZ$ (that is, a vector bundle over the universal complex $n$-manifold) we obtain a family over $\cX$ whose fiber over a complex manifold $X$ is the $\beta-\gamma$ system on $X$ twisted by the tensor bundle $\cZ$.
For example, if $\cZ = \cT$ is the universal tangent bundle, then the theory over $X$ is the $\beta-\gamma$ system on $X$ twisted by $Z = \T_{X}$.

Working with families of theories over \textit{formal} derived stacks, we can be more explicit.
A symmetry for a formal derived stack described by a dg Lie algebra $\cL$ simply amounts to an infinitesimal $\cL$-symmetry which, in field theory, we assume is prescribed in terms of local currents.
We recall precise definitions below.
For a quantum field theory with a $\cL$-symmetry, the work \cite{fact2} formulates a version of Noether's theorem in terms of factorization algebras.
The result says that for a theory with a $\cL$ action, and factorization algebra of observables $\Obs$, there is a map
of factorization algebras
\begin{equation}\label{eqn:noether1}
C_{\bu}^{\theta}(\cL_{c}) \to \Obs
\end{equation}
where, we recall from section \ref{s:enveloping} that $C_{\bu}^{\theta}(\cL_{c})$ is the $\theta$-twisted enveloping factorization algebra of $\cL$.
The twist~$\theta$ can be understood as the t'Hooft anomaly associated to the symmetry and is explicitly an obstruction to solve the so called (equivariant) master equation in the BV formalism.

Now, let $X$ be a complex manifold.
The dg Lie algebra $\cL= \cT(X) = \Omega^{0,\bu}(X, \T_{X})$ describes the formal moduli problem of deformations of complex structure.
In other words, it is the formal neighborhood of $X$ inside of the moduli of all complex manifolds.
A $\cT$-symmetry amounts to prescribing a holomorphic version of the stress tensor in ordinary quantum field theory.
In complex dimension one, this is precisely the chiral half of a conformal structure on a theory.
The anomaly $\theta$ accounts for, at the level of the moduli space, the Beilinson--Schechtman line bundle over the moduli
space of Riemann surfaces \cite{BS}.
In this setting, we will find a presentation for this line bundle as a local cocycle for the local Lie algebra $\cT$.


As a conequence of the factorization algebra enhancement of Noether's theorem of \cite{fact2}, we obtain the following linking back to our main definition of a \textit{holomorphic structure}.
\begin{prop}
  A $\cT_{X}$-symmetry on a theory whose underlying factorization algebra of quantum observables is $\Obs$ defines a holomorphic structure on $\Obs$, see definition \ref{dfn:hol}.
  The central charge $\bc$ is determined by the obstruction for this $\cT_{X}$ symmetry to solve the equivariant master equation.
\end{prop}

The primary goal of this section is to present examples of holomorphic structures on free theories in various dimensions.

\subsection{Equivariant master equation}

We discuss the precise notion of a family of theories over the formal derived stack $B \cL$, where $\cL$ is a local dg
Lie algebra---this is, equivalently, the notion of a (local) $\cL$-symmetry.

Let's start with an approach to symmetry in shifted Poisson geometry.
Suppose that $V$ is a graded vector space and $\eta \in \wedge^2 V^*$.
Dually, we can think of $\eta$ as a Poisson bivector for the vector space $E = V^*$.
The bivector $\eta$ induces a Poisson bracket on $\Sym (V) = \cO(E)$.
A strict Hamiltonian action of a Lie algebra $\lie{g}$ on $E$ is a map of Lie algebras $\mu \colon \lie{g} \to \cO(E)$.
By Koszul duality between Lie and commutative algebras, see for example \cite{LurieDAGX}, we can view such a linear map $\mu$ as an element of the dg Lie algebra
$C^\bu(\lie{g}) \otimes \cO(E)$ of cohomological degree one satisfying the Mauer--Cartan equation.
More generally, $L_\infty$ maps $\lie{g} \rightsquigarrow \cO(E)$ are \textit{defined} to be the Maurer--Cartan elements in
the dg Lie algebra $C^\bu(\lie{g}) \otimes \cO(E)$.

Now, suppose that $\lie{g}$ is a dg Lie algebra and $(E,Q)$ is a cochain complex equipped with a $Q$-closed bivector
$\eta \in \wedge^2 E$ of cohomological degree zero.
The Chevalley--Eilenberg complex $C^\bu(\lie{g})$ is a commutative dg algebra and so the Poisson bracket endows $C^\bu(\lie{g}) \otimes \cO(E)$ with the structure of a dg Lie algebra; the differential is the sum of $\d_{CE}$ and $Q$.
A (weak) \textit{Hamiltonian action} is, more generally, a Maurer--Cartan element of the dg Lie algebra $C^\bu(\lie{g}) \otimes \cO(E)$.
That is, an element of degree one
\begin{equation}
\mu \in \left[C^\bu(\lie{g}) \otimes \cO(E)\right]^1
\end{equation}
such that
\begin{equation}
\d_{CE} \mu + Q \mu + \frac12 \{\mu,\mu\} = 0 .
\end{equation}
This generalizes the notion of \textit{strict} Hamiltonian action to allow for $L_\infty$ maps $\mu \colon \lie{g}
\rightsquigarrow \cO(E)$.

Now, we move to our setting where $\eta$ carries cohomological degree $+1$.
Let $(E,Q)$ and $\lie{g}$ be the same as above. 
This time, $\eta$ equips $\cO(E)[-1]$ with the
structure of a dg Lie algebra (notice the shift by one).
By definition, a $\lie{g}$-symmetry on the $0$-dimensional free quantm field theory defined by $(E,Q,\eta)$ is a
Maurer--Cartan element in the dg Lie algebra $C^\bu(\lie{g}) \otimes \cO(E)[-1]$.
Note that the underlying graded Lie algebra underlying this dg Lie algebra is $\cO(\lie{g}[1] \oplus E)$.

The work of \cite{fact2} formulates the general notion of symmetry in perturbative quantum field theory.
This amounts, in part, to replacing $(E,Q)$ with a complex of vector bundles, whose sections are the space of fields of the theory, and $\lie{g}$ with a local Lie algebra,
which we denote by $\cL = \Gamma (L)$.
%Now, we replace $V$ by a complex of vector bundles.
We will assume for simplicity that $\eta$ is of the form
\begin{equation}\label{eqn:etaD}
\eta(\phi) = \frac12 \int \phi D \phi
\end{equation}
where $D \colon V \to V^![-1]$ is an elliptic differential operator of cohomological degree $-1$.
Let $E = V^!$ be the $!$-dual bundle.
Then, we have seen that the bracket defined using $\eta$ endows the complex of local functionals $\oloc(E)[-1]$ with
the structure of a dg Lie algebra (after shifting to account for the degree of $\eta$).
This bracket extends to a bracket on $\oloc(L[1] \oplus E)[-1]$. 
Combined with the total differential $\d_{CE} + Q$, where $\d_{CE}$ is the Chevalley--Eilenberg differential associated
to $\cL$, we obtain the structure of a dg Lie algebra that we denote 
\begin{equation}\label{eq:def}
    \op{Def}(L,E) \define \big(\oloc(L[1] \oplus E)[-1] , d_{CE} + Q, \{-,-\}_\eta \big)
\end{equation}
By definition, a \defterm{local Hamiltonian action} of $\cL$ on the free theory $(E,Q,\eta)$ is a Maurer--Cartan element
$J \in \op{Def}(L,E)^1$ which is non-constant in the fields (sections of $E$).

\begin{thm}\ref{thm:freenoether}
Suppose that $(V,Q)$ is an elliptic complex of vector bundles on $M$ and that $\eta \in \oloc(V)$ is a local functional of the form \eqref{eqn:etaD}.
Suppose that $J^{cl}$ is a local Hamiltonian action of a local Lie algebra $\cL$ on $E = V^!$.
Then, there is a local cohomology class
\begin{equation}
\Theta \in H^1_{loc}(\cL)
\end{equation}
together with a map of factorization algebras
\begin{equation}
J \colon \mathbf{U}_{\Theta} \cL \to \cW_{V,\eta}
\end{equation}
such that $\op{gr} J = J^{cl}$.
\end{thm}

In the case that $\cL = \cT_X$ is the local Lie algebra of holomorphic vector fields on a complex manifold $X$, then we
will denote the local current $J$ by the symbol $\sfT$ to emphasize its role as the "holomorphic stress tensor" of the
theory.



%Suppose that $P$ is a $(-1)$-shifted Poisson dg vector space.
%Then we have seen that the (BV) bracket endows $\cO(P)[-1]$ with the structure of a dg Lie algebra (generally this is just an $L_\infty$ algebra if the Poisson structure $\pi$ has components of weight $\geq 3$).
%A strict Hamiltonian action of a dg Lie algebra $\lie{g}$ on $P$ is the data of a map of dg Lie algebra
%\begin{equation}\label{eqn:moment}
%T \colon \lie{g} \to \cO(P)[-1]\footnote{Alternatively, this can be thought of as arising from a shifted version of the moment map $P \to \lie{g}^*[-1]$, where the $(-1)$-shifted Poisson structure on $\lie{g}^*[-1]$ is, like the Kostant--Kirillov Poisson structure, determined by the Lie bracket of $\lie{g}$.}.
%\end{equation}
%The (BV) bracket is a map of dg Lie algebras $\{-,-\} \colon \cO(P)[-1] \to \op{Vect}(P)$; hence a Hamiltonian action defines, in particular, an infinitesimal $\lie{g}$-action on $P$.
%
%Hamiltonian actions of the above type can be classified as follows.
%Recall that the Chevalley--Eilenberg cochain complex $C^\bu(\lie{g})$ is a commutative dg algebra whose cohomology is the Lie algebra cohomology of $\lie{g}$ (with trivial coefficients).
%Consider the dg Lie algebra
%\begin{equation}\label{eqn:babycme}
%C^\bu(\lie{g}) \otimes \cO(P)[-1] 
%\end{equation}
%where the bracket uses the product on the Chevalley--Eilenberg complex and the bracket on $\cO(P)$.
%Hamiltonian actions \eqref{eqn:moment} define Maurer--Cartan elements in this dg Lie algebra.
%More generally, we define a \defterm{Hamiltonian action} of $\lie{g}$ on the $(-1)$-shifted Poisson space $P$ to be a Maurer--Cartan element in this dg Lie algebra.
%Directly in terms of the Lie algebra $\lie{g}$ these correspond to $L_\infty$ morphisms from $\lie{g}$ to $\cO(P)[-1]$.
%Strict Hamiltonian actions are those which are linear in $\lie{g}$.
%The Maurer--Cartan element equation for \eqref{eqn:babycme} is
%\begin{equation}
%\d_{CE} J + \d_{P} J + \frac12 \{J,J\}  = 0
%\end{equation}
%where $\d_{CE}$ is the Chevalley--Eilenberg differential for $\lie{g}$, $\d_P$ is the internal differential to $P$, and $\{-,-\}$ is the $(-1)$-shifted Poisson bracket on $\cO(P)$.\footnote{If $\pi$ has components of higher weight then \eqref{eqn:babycme} is only an $L_\infty$ algebra and the MC equation will also involve the higher brackets.}

\subsection{The $\beta-\gamma$ system}

Fix a complex dimension $n$ and suppose that $W$ is a natural holomorphic vector bundle functor on complex
$n$-manifolds \cite{MK}.
This is a functor from the category of all $n$-dimensional complex manifolds with morphisms given by local holomorphic
diffeomorphisms to the category of vector bundles.
Examples include the trivial bundle, the (co)tangent bundle, tensor bundles, and quotients of such.

Fix a complex manifold $X$ and let $W_X$ denote the corresponding holomorphic vector bundle on $X$.
The naturality assumption implies that the space of sections $\Gamma(W)$ is a module for the Lie algebra of holomorphic vector
fields; the action is by Lie derivative.
This furthermore extends to an action of the dg Lie algebra $\Omega^{0,\bu}(X,\T_X)$ on the cochain complex
$\Omega^{0,\bu}(X,W_X)$ which we denote 
\begin{equation}\label{}
  (\mu, \gamma) \mapsto L_\mu \gamma 
\end{equation}
where $\mu \in \Omega^{0,\bu}(X,\T_X)$ and $\gamma \in \Omega^{0,\bu}(X,W_X)$.

The free theory we consider is the $\beta-\gamma$ system on $X$ with values in the natural bundle $W_X$.
This means that $\gamma \in \Omega^{0,\bu}(X,W_X)$ and $\beta \in \Omega^{0,\bu}(X,W_X^!)[n-1]$.
Notice that the (free) classical action $\int_X \beta \dbar \gamma$ is manifestly invariant under the action of
holomorphic vector fields.

There is a local Hamiltonian which encodes this symmetry.
For $\mu \in \cT_X$ define the local functional 
\begin{equation}\label{}
  \sfT(\mu) = \int_X \beta L_X \gamma .
\end{equation}

\begin{prop}
  The local functional $\sfT \in \oloc(\cT[1] \oplus E)$ satisfies the Maurer--Cartan equation and hence defines a
  local Hamiltonian symmetry of holomorphic vector fields on the $\beta-\gamma$ system on $X$ with values in the
  natural bundle $W_X$.
\end{prop}

\begin{proof}
We need to check the equation
\begin{equation}\label{eqn:mcbg}
  \d_{CE} \sfT + \dbar \sfT + \frac12 \{\sfT, \sfT\} = 0 .
\end{equation}
It is clear that $\dbar \sfT = 0$.
The Chevalley--Eilenberg differential applied to $\sfT$ gives the local functional, which is quadratic in vector fields: 
\begin{equation}\label{}
  (\d_{CE} \sT)(\mu,\mu') = -\sfT([\mu,\mu'])= - \int_X \beta L_{[\mu,\mu']} \gamma . 
\end{equation}
On the other hand, we have 
\begin{equation}\label{}
  \frac12 \{\sfT,\sfT\} (\mu,\mu') = \{\sfT(\mu),\sfT(\mu')\} = \int \beta (L_\mu L_{\mu'} - (-1)^{|\mu'|} L_{\mu'} L_\mu)
\gamma = \sfT([\mu,\mu']).
\end{equation}
\end{proof}

We now turn to quantization.
By \ref{thm:freenoether}, we know that there exists a higher central charge $\bc \in H^{2n+2}(BU(n))$ for $\cT$ together with a homomorphism of
factorization algebras 
\begin{equation}\label{}
  \cV\op{ir}_{\bc} \to \beta\gamma(W)   
\end{equation}

\subsection{The chiral boson}


\begin{thm}
Let $X$ be a complex manifold of dimension $2n+1$ and consider the higher dimensional chiral boson factorization algebra $\cB$ on $X$.
The current $\frac12 \int \alpha \iota_X \alpha$ extends to a map of factorization algebras
\begin{equation}
\cV\op{ir}_{\mathbf{c}} \to \cB 
\end{equation}
where
\begin{equation}
\mathbf{c} \define \sum^{n}_{k = 0} (-1)^k \op{Td} \cdot \op{ch}(\wedge^k T)|_{4n+4} \in H^{4n+4}(BU(2n+1)) .
\end{equation}
\end{thm}
In this section, we fix a complex manifold $X$ of dimension $2n+1$ on which all objects are defined.
For $\mu \in \cT$ define the local functional
\begin{equation}
T(\mu) \in \oloc(\fields)
\end{equation}
by the formula
\begin{equation}
T(\mu) = \sum_{k \geq 1} f(k) \int \alpha \iota_\mu^k \alpha .
\end{equation}

\begin{lem}
The functional
\begin{equation}
T \in \op{Act}(\cT,\fields)^1 
\end{equation}
satisfies the classical master equation
\begin{equation}
\d_{\cT} T + \del T + \frac12 \{T,T\} = 0 .
\end{equation}
\end{lem}
\begin{proof}
First we compute
\begin{equation}
\d_{\cT} T = - \sum_{k \geq 1} k f(k) \int \alpha \iota_{[\mu,\mu]} \iota^{k-1}_\mu \alpha 
\end{equation}
and
\begin{equation}
\del T = \sum_{k \geq 1} \int \alpha \iota_\mu^k \del \alpha .
\end{equation}

Next, we compute
\begin{equation}
\{\int \alpha \iota_{\mu} \alpha, \int \alpha \iota_{\mu'} \alpha\} = \int \del 
\end{equation}
\end{proof}

\subsection{Modified stress tensor}

\section{Examples}

\subsection{The ordinary Segal--Sugawara construction}

The most general stress tensor is
\begin{equation}
T_a(X) = \int \alpha \iota_X \alpha + \hbar a \int \alpha (J X)  
\end{equation}

The central charge is $c = 1 + \hbar^2 \frac{1}{24} a^2$.

\subsection{Coupling in topological string theory}

The most general stress tensor is
\begin{equation}
T_{\ba} (X) = \int \alpha \iota_X \alpha + a_1 \int \alpha \op{Tr}(J X \del J X) + a_2 \int \alpha \op{Tr}(J X) \op{Tr}(\del J X) .
\end{equation}

\subsection{Coupling in type IIB string theory}

\section{Higher vertex algebras}

\subsection{The higher Heisenberg algebra}

We will use the explicit commutative dg algebra model $\sfA$ for $\R \Gamma(\mathring{D}^{2n+1}, \cO)$ where $\mathring{D}^{2n+1}$ is the punctured formal $(2n+1)$-disk which was introduces in \cite{FHK} and recalled in appendix \ref{appx:model}.
Denote the differential on this dg model by $\dbar$.
In appendix \ref{appx:model} we have also introduced dg $\sfA$-modules $\sfA^{(p)}$ for $\R \Gamma(\mathring{D}^{2n+1}, \Omega^{p})$ where $\Omega^{p}$ is the sheaf of $p$-forms on the formal disk.
We set $\sfA = \sfA^{(0)}$.
Let $\del \colon \sfA^{(p)} \to \sfA^{(p+1)}$ be the formal holomorphic de Rham operator.

The formal $S^{4n+1}$-residue is a cochain map
\begin{equation}\label{eqn:residue}
  \oint \colon \sfA^{(2n+1)} \to \C[-2n+2] .
\end{equation}


Consider the following abelian dg Lie algebra
\begin{equation}\label{eqn:f}
\mathsf{f} \define \sfA[2n] \xto{\del} \sfA^{(1)} [2n-1] \to \cdots \to \sfA^{(n)}[n]
\end{equation}
concentrated in degrees $-2n$ up to $1$.
Notice that this is the totalization of a double complex; we have left the $\dbar$-degree and operator intrinsic to this expression.

The higher Heisenberg dg Lie algebra $\mathsf{h}$ is the dg Lie algebra central extension
\begin{equation}\label{eqn:higherheis}
0 \to \C \to \mathsf{h} \to \mathsf{f} \to 0
\end{equation}
determined by the degree two cocycle
\begin{equation}\label{eqn:cocycleheis}
\oint \alpha \del \beta
\end{equation}
where $\alpha, \beta \in \mathsf{f}$.
Notice that this expression is only nonzero on the holomorphic $n$-form components of $\alpha$ and $\beta$.

We focus on the case $n=3$ from now on in this section.

Note that in this case, $\mathsf{h}$ is concetrated in degrees $-2$ up to $1$.
In fact, since $\mathsf{h}$ is formal, we might as well replace it with its cohomology $\lie{h}' \define H^{\bu}(\mathsf{h})$.
This cohomology is concentrated in degrees $-2,-1,0,1$.
In degree $-2$ the cohomology is one-dimensional given by the constant functions on the formal disk and this summand appears trivially in all Lie brackets.
We will mostly consider the graded Lie subalgebra $\lie{h} \subset \lie{h}'$ which agrees with $\lie{h}'$ in degrees $-1,0,1$.
In degree $-1$ we have
\begin{equation}\label{eqn:deg-1}
\lie{h}^{-1} = \Omega^{1} (D^{3}) \slash \del \cO(D^{3})
\end{equation}
The degree zero part of $\lie{h}$ is one-dimensional $\lie{h}^{0} = \C K$, spanned by the central element $K$.
The degree one part is, by formal Serre duality, given by
\begin{equation}\label{eqn:formalserre}
  \lie{h}^{1} = H^{2}(\mathring{D}^{3}, \Omega^{1} \slash \del \cO) \simeq \left(\Omega^{2,cl}(D^{3})\right)^{\vee}
\end{equation}
where $\Omega^{2,cl}(D^{3})$ is the space of closed two-forms on the formal three-disk.


%In fact, we can simplify this model slightly.
%Let $\Omega^{1} \slash \d \cO$ denote the sheaf of one-forms modulo the sheaf of exact one-forms.
%We can then consider the Joaunalou model $\sfA^{(0 \to 1)}$ for $\R \Gamma (\mathring{D}^{3}, \Omega^{1} \slash \d \cO)$.
%It has the following explicit form.
%
%Recall that in degree zero $\sfA^{0} = \sfA^{(0),0}$ is the localized algebra $\C[[z]][z^{*}][(z z^{*})^{-1}]$.
%Similarly, from the $\cO = \C[[z]]$-module $\Omega^{1} \slash \d \cO$ we can localize to get the $\sfA^{0}$-module $\Omega^{1} \slash \d \cO [z^{*}][(z z^{*})^{-1}]$ which we denote by $\sfA^{(0 \to 1),0}$.
%
%For $0 \leq m \leq 2$ we identify $\sfA^{(0 \to 1),m}$ with the vector space formed by expressions
%\begin{equation}\label{eqn:forms}
%\alpha = \sum_{1 \leq i_{1} \leq \cdots \leq i_{m} \leq 3} \alpha_{i_{1}\cdots i_{m}}(z, z^{*}) \d z_{i_{1}}^{*} \cdots \d z_{i_{m}}^{*}
%\end{equation}
%where $\alpha_{i_{1}\cdots i_{m}} \in \sfA^{(0 \to 1),0}$ and
%\begin{itemize}
  %\item[(i)] $\alpha$ is homogenous in the starred variables $z_{i}^{*}, \d z_{i}^{*}$ of total degree zero.
  %\item[(ii)] The contraction of $\alpha$ with the Euler vector field $Eu^{*} = \sum_{i} z_{i}^{*} \frac{\del}{\del z_{i}^{*}}$ vanishes, $i_{Eu^{*}} \alpha = 0$.
%\end{itemize}
%The differential on $\sfA_{m}$ is simply the $\dbar$ operator $\dbar = \sum_{i} \d z_{i}^{*} \frac{\del}{\del z_{i}^{*}}$.
%
%This dg $\sfA$-module $\sfA^{(0 \to 1)}$ is concentrated in degrees $0,1,2$ (just like $\sfA$).
%In particular, $\lie{f} = \sfA^{(0\to 1)}[1]$ can be considered as an abelian dg Lie algebra concentrated in degrees $-1,0,1$.
%The same formula as above defines the one-dimensional central extension $\lie{h}$ of $\lie{f}$.
%
%Let $\lie{h}_{-} = H^{1}(\lie{f}_{-}^{1}) [-1] = H^{2}(\sfA^{(0 \to 1)})[-1]$ which is a vector space concentrated in degree $+1$.
%
%\begin{lem}
  %There is an isomorphism $\lie{h}_{-} \simeq \left(\Omega^{2,cl}(D^{3})\right)^{\vee}[-1]$.
%\end{lem}
%
%Let $\lie{h}_{+} = \ker (\lie{h} \to \lie{h}_{-})$ where the map is simply the identity in degree $+1$.
%Note that $\lie{h}_{+}$ is concentrated in degrees $-1$ and $0$.
%There is, of course, a splitting of cochain complexes $\lie{h}= \lie{h}_{-} \oplus \lie{h}_{+}$, but this splitting does not respect the Lie bracket.
%In fact, $\lie{h}_{+}$ is a sub dg Lie algebra.
%
%\brian{below is the version where I take cohomology}

Define the abelian subalgebras $\lie{h}_{+} = \lie{h}^{-1} \oplus \C K$ and $\lie{h}_{-} = \lie{h}^{1}$ so that
\begin{equation}\label{}
  \lie{h} = \lie{h}_{+} \oplus \lie{h}_{-} \simeq \left(\frac{\Omega^{1}(D^{3})}{\del \cO(D^{3})} [1] \oplus \C K\right) \oplus \left(\Omega^{2,cl}(D^{3})\right)^{\vee} [-1] .
\end{equation}


\subsection{The vacuum module}

Let $\C_{K=1}$ denote the one-dimensional $\lie{h}_{+}$-module (concentrated in degree zero) defined by the rules:
\begin{itemize}
  \item $K \in \lie{h}_{+}^{0}$ acts by the identity.
  \item By degree reasons, $\lie{h}_{+}^{-1} = \Omega^{1} \slash \del \cO$ necessarily acts trivially on $\C_{K = 1}$.
\end{itemize}

Define the dg $\lie{h}$-module
\begin{equation}\label{eqn:module}
  \pi \define U(\lie{h}) \otimes_{U(\lie{h}_{+})} \C_{K=1}
\end{equation}

By construction, we see that there is an isomorphism of graded vector spaces
\begin{equation}\label{eqn:isomorphism?}
\pi \simeq \C[\lie{h}_{-}] \simeq \op{Sym} \left( \Omega^{2,cl}(D^{3})^{\vee}[-1] \right) .
\end{equation}
In other words, as a vector space $\pi$ is the free graded commutative algebra on the vector space $H^{2}\sfA^{(0 \to 1),2}[-1] \simeq (\Omega^{2,cl}(D))^{\vee}[-1]$ (concentrated in degree one).

We can describe the $\lie{h}$-module structure explicitly.
In degree $-1$ we have $\lie{h}^{-1} = \Omega^{1}(D)\slash \del \cO(D^{3})$.
Recall that we have used the canonical isomorphism
\begin{equation}\label{eqn:serre}
\Omega^{2,cl}(D^{3})^{\vee} \simeq H^{2}(\mathring{D}^{3}, \Omega^{1} \slash \del \cO) .
\end{equation}
granted by the higher residue pairing.
Given a class $[\alpha] \in \Omega^{1}(D)\slash \del \cO$, and a linear generator $\beta \in \Omega^{2,cl}(D^{3})^{\vee} \simeq H^{2}(D, \Omega^{1} \slash \del \cO)$ we define
\begin{equation}\label{eqn:alphaaction}
\alpha \cdot \beta = \oint \alpha \del \beta \in \C .
\end{equation}
We then extend this, as a graded derivation on $\pi$ of cohomological degree $-1$, to a linear map that we denote
\begin{equation}\label{eqn:derivation}
  \frac{\del}{\del \alpha} \colon \pi \to \pi [-1], \quad \alpha \in \lie{h}^{-1} = \frac{\Omega^{1}(D^{3})}{\del \cO(D^{3})} .
\end{equation}


In degree zero, the central element $K \in \lie{h}_{+}$ acts on $\pi$ by the identity.

Finally, in degree $-1$ we have $\lie{h}^{1}= \lie{h}_{-} = \left(\Omega^{2,cl}(D^{3})\right)^{\vee}$.
Such elements act simply by multiplication (this is sensible since $\lie{h}_{-}$ is concentrated in degree $+1$ just like the generators of $\pi$).
Thus, we obtain linear maps
\begin{equation}\label{eqn:created}
  \alpha \cdot \colon \pi \to \pi [1] , \quad \alpha \in \lie{h}^{1} = \left(\Omega^{2,cl}(D^{3})\right)^{\vee} .
\end{equation}

\subsection{Explicit generators}

We provide explicit generators and relations for the algebras and modules of the previous section.

For $k_{1},k_{2},k_{3} \geq 0$, and $i=1,2,3$, let
\begin{equation}\label{eqn:positive}
  b_{i}[\bm] = b_{i}[m_{1},m_{2},m_{3}] \in \lie{h}_{+} = \Omega^{1}(D^{3})/ \del \cO(D^{3})
\end{equation}
be the equivalence class of the one-form $z_{1}^{m_{1}} z_{2}^{m_{2}}z_{3}^{m_{3}} \d z_{i}$ on the formal disk.
By definition, $b_{i}[\bm]$ is of cohomological degree $-1$.
Similarly, let
\begin{equation}\label{eqn:negative}
  b_{i}[-\bm-1] = b_{i}[-m_{1}-1,-m_{2}-1,-m_{3}-1] \in \lie{h}_{-} = \Omega^{2,cl}(D^{3})^{\vee}
\end{equation}
be the element linearly dual to the closed two-form $\del(z_{1}^{m_{1}} z_{2}^{m_{2}}z_{3}^{m_{3}} \d z_{i})$ on the formal disk.
The elements $b_{i}[-\bm-1]$ are of cohomological degree $+1$.
Thus, $\{b_{i}[\bm], K\}$ where $i=1,2,3$ and $\bm \geq 0$ (in the sense that $m_{j} \geq 0$ for $j=1,2,3$).

The Lie bracket for $\lie{h}$, in terms of this basis, is
\begin{align*}
  \left[b_{i}[\bm], b_{j}[-\bn-1]\right] & = \sum_{k=1}^{3} \eps_{ijk} n_{k} \delta_{\bm+e_{k}, \bn} K \\
  \left[b_{i}[\bm], b_{j}[\bn]\right] & = 0 \\
  \left[b_{i}[-\bm-1], b_{j}[-\bn-1]\right] & = 0 .
\end{align*}
In this expression, $e_{j}$ stands for the length three multi-index with $1$ in entry $j$ and $0$'s elsewhere and $\delta_{\bk, \bl} = \delta_{k_{1},l_{1}}\delta_{k_{2},l_{2}} \delta_{k_{3},l_{3}}$.

As a vector space, the module $\pi$ is the graded polynomial algebra on the cohomological degree one generators $b_{i}[-\bm-1]$ for $i=1,2,3$ and $\bm \geq 0$.
For example, in cohomological degree zero $\pi^{0} = \C$, spanned by the unit.
In degree one, we have $\pi^{1}=\lie{h}_{-}$ spanned by $\{b_{i}[-\bm-1]\}$.
In degree two, $\pi^{2}$ is spanned by antisymmetric expressions of the form
\begin{equation}\label{eqn:deg2}
b_{i}[-\bm-1,\bn-1] b_{j}[-\br-1,\bs-1]
\end{equation}
subject to the relations
\begin{equation}\label{}
b_{i}[-\bm-1,\bn-1] b_{j}[-\br-1,\bs-1] = b_{j}[-\br-1,\bs-1] b_{i}[-\bm-1,\bn-1]  .
\end{equation}

The $\lie{h}$-module structure on $\pi$ is as follows.
For $\bn \geq 0$ the elements $b_{j}[-\bn-1]$ act on $\pi$ by left multiplication.
These are the ``creation'' operators.
The element $K$ acts by the identity.
For $\bn > 0$ (meaning $n_{j} \geq 0$ and $n_{1}+n_{2}+n_{3} > 0$) the elements $b_{j}[\bn]$ acts by the derivation\begin{equation}\label{eqn:derivationexplicit}
  \frac{\del}{\del b_{j}[-\bn]}.
\end{equation}
These are the ``annihilation'' operators.
Finally, $b_{j}[0,0,0]$ acts by zero for $j=1,2,3$.

%In conformal triweight $(0,0,0)$ we have just the unit $1 \in \C$.
%In conformal triweight $e_{j}$ we have $b_{j}[-1,-1,-1]$, $j=1,2,3$.
%In conformal triweight

\subsection{Segal-Sugwara, locally}

Consider the following vectors
\begin{equation}\label{}
\omega_{i} = \sum_{j<k} \eps_{ijk} b_{j}[-1] b_{k}[-1] , \quad i=1,2,3
\end{equation}
where $b_{j}[-1] = b_{j}[-1,-1,-1]$.

\subsection{Higher vertex algebra structure}


\newpage

\appendix

\section{A dg model for punctured affine space} \label{appx:A}

In this appendix we recall a commutative dg model for punctured affine space $\A^n - \{0\}$ defined in \cite{FHK}.
Let $z_i,z_i^*$, $i=1,\ldots,n$ be variables of cohomological degree zero and let $z z^* \define \sum_{i=1}^n z_i z_i^*$.
Let $\sfA_{[n]}^\bu$ be the commutative graded algebra generated over the localized ring
\begin{equation}
\C[z_i,z_i^*] [(z z^*)^{-1}]
\end{equation}
by degree one generators $\d z_i^*$, $i=1,\ldots,n$ subject to the conditions
\begin{itemize}
\item[(i)] Give $z^*_i, \d z^*_i$ weight $+1$ for $i=1,\ldots,n$.
An element $\alpha \in \sfA_{[n]}$ is required to have total weight zero.
\item[(ii)] Let $Eu^* = \sum_i z_i^* \del_{z_i^*}$, then an element $\alpha \in \sfA_{[n]}$ is required to satisfy $\iota_{Eu^*} \alpha = 0$.
\end{itemize}

Let $\dbar = \sum_i \d z_i^* \del_{z_i^*}$ be the degree one operator acting on $A_{[n]}$.
Then $\dbar^2 = 0$ and $(\sfA_{[n]},\dbar)$ is a commutative dg algebra.

\begin{thm}[\cite{FHK}]
The dg algebra $(\sfA_{[n]},\d)$ is a dg model for $\R \Gamma(\A^n - \{0\}, \cO)$.
\end{thm}

We observe that any element of the form $f(z,z^*) \d^n z^* = f(z,z^*) \d z_1^* \cdots \d z_n^*$ does not satisfy the conditions (i),(ii).
In particular, we see that $\sfA_{[n]}$ is concentrated in cohomological degrees $0,\ldots,n-1$.
As a model for punctured affine space, the cohomology of $\sfA$ is concentrated in degrees zero and $n-1$.
In degree zero the cohomology is simply globally defined algebraic functions on punctured affine space, which by Hartog's theorem is
\begin{equation}
H^0(\sfA_{[n]}) = \C[z_1,\ldots,z_n] .
\end{equation}
Next, define
\begin{equation}
\omega_{BM} = \# \sum_{i=1}^n (-1)^{i+1} \frac{z_i^* \d z_1^* \cdots \Hat{\d z_i^*} \cdots \d z_n^* }{(z z^*)^n}
\end{equation}
be the Bochner--Martenilli kernel.
One can check that $\omega_{BM}$ is a degree $(n-1)$ element of $\sfA_{[n]}$
Then a presentation for the degree $(n-1)$ cohomology is 
\begin{equation}
H^{n-1}(\sfA_{[n]}) \cong \C[\del_{z_1},\ldots,\del_{z_n}] \omega_{BM} .
\end{equation}

In the main text, we find it convenient to introduce the notation
\begin{equation}
\lambda_i \define \frac{\zbar_i}{(zz^*)} .
\end{equation}
When $n=1$ this is simpy $\lambda = \frac{1}{z}$. 

There is an embedding of commutative dg algebras \cite{FHK}:
\begin{equation}
j \colon \sfA \hookrightarrow \Omega^{0,\bu}(\C^n - \{0\})
\end{equation}
which sends $z_i \mapsto z_i$, $z_i^* \mapsto \zbar_i$, $\d z^*_i \mapsto \d \zbar_i$, which clearly interchanges the algebraic $\dbar$-operator with the usual one.
This embedding becomes a dense inclusion at the level of cohomology.
In the main text, we do not distinguish the polynomial variable $z_i^*$ with the anti-holomorphic coordinate $\zbar_i$.

We will utilize a higher residue map.
Let 
\begin{equation}
\op{Res} \left(- \d^n z\right) \colon \sfA \to \C[-n+1]
\end{equation}
be defined by the projection onto the component of $\omega_{BM} \in \sfA_{[n]}$, which is of degree $n-1$.
As observed in \cite{FHK}, one has compatibility with usual integration over any $(2n-1)$-sphere centered at the origin as
\begin{equation}
\Res_z(\alpha \d^n z) = \oint_{S^{2n-1}} j(\alpha) d^n z ,
\end{equation}
for $\alpha \in \sfA_{[n]}$.

\subsection{Three-dimensional example}
In this appendix we unpack the model above in the special case of the three-dimensional punctured affine space to come up with a description via generators and relations.

Let $\sfA'$ be the graded polynomial algebra
\begin{equation}
\C[z_i, \lambda_j, \mu_k]
\end{equation}
where $i,j,k=1,2,3$ and the cohomological degrees of the generators are:
\begin{equation}
\op{deg}(z_i) = \op{deg}(\lambda_j) = 0, \quad \op{deg}(\mu_i) = 1 .
\end{equation}
Define the degree $2$ element
\begin{equation}
\omega \define \sum_{ijk} \eps_{ijk} z_i \mu_j \mu_k = z_1 \mu_2 \mu_3 - z_2 \mu_1 \mu_3 + z_3 \mu_1 \mu_2 .
\end{equation}
Define the differential $\d$ on $\sfA'$ on generators as follows
\begin{align*}
\d z_i & = 0 \\
\d \lambda_i & = - \sum_{jk} \eps_{ijk} z_j \mu_k \\
\d \mu_i & = 2 z_i \omega .
\end{align*}
This is a differential, $\d^2 = 0$, since $\eps_{ijk} z_j z_k = 0$ for any $i=1,2,3$.
Thus, $(\sfA', \d)$ is a commutative dg algebra.
Notice that $\d (\sum_i z_i \lambda_i) = 0$ and $\d (\mu_i \mu_j) = \d (\sum_k \eps_{ijk} \lambda_k \omega)$.

We consider the quotient $\sfA$ of the graded algebra $\sfA'$ by the ideal generated by the relations
\begin{enumerate}
\item $\sum_i \lambda_i z_i = 1$.
\item $\sum_i \lambda_i \mu_i = 0$.
\item $\mu_1 \mu_2 \mu_3 = 0$. 
\end{enumerate}

\begin{lem}
The differential preserves the relations (1),(2), and (3).
Thus, $(\sfA, \d)$ is a commutative dg algebra which is concentrated in degrees zero, one, and two.
\end{lem}
\begin{proof}
The differential preserves (1) since $\sum_{jk} \eps_{ijk} z_j z_k = 0$.
Now
\begin{align*}
\dbar( \sum_i \lambda_i \mu_i ) & = - (\sum_{ijk} \eps_{ijk} z_j \mu_k \mu_i) + 2\sum_i \lambda_i z_i \omega \\
& = 
\end{align*} 
\end{proof}

%\begin{lem}
%In the graded algebra $\sfA$ one has the relations $\mu_1 \mu_2 \mu_3 = 0$ and 
%\begin{equation}
%\sum_{ijk} \eps_{ijk} \mu_i \mu_j z_k = \omega .
%\end{equation}
%\end{lem}
%\begin{proof}
%From relation (3) we have $\mu_1 \mu_2 \mu_3 = \lambda_i \mu_i \omega$ for $i=1,2,3$.
%Thus using relation (2) we have $3 \mu_1 \mu_2 \mu_3 = \sum_i \lambda_i \mu_i \omega = 0$.
%Similarly, using (3) we have 
%\begin{equation}
%\sum_{ijk} \mu_i \mu_j z_k = \sum_k \lambda_k z_k \omega = \omega .
%\end{equation}
%\end{proof}

%It follows from this lemma that the graded algebra $\sfA$ is concentrated in degrees $0,1,2$. 
%Indeed from the above lemma we see
%\begin{equation}
%\omega^2 = \left(\sum_{ijk} \eps_{ijk} \mu_i \mu_j z_k \right)^2 = 0
%\end{equation}
%and
%\begin{equation}
%\mu_i \omega = 0.
%\end{equation}

\begin{prop}
The dg algebra $(\sfA,\d)$ is a dg model for $\R \Gamma(\A^3 - 0, \cO)$.
\end{prop}
\begin{proof}
This model is isomorphic to $\sfA_{[n]}$ recollected in the previous section by the isomorphism
\begin{align*}
\lambda_i & \leftrightarrow z_i^* (zz^*)^{-1} \\
\mu_i & \leftrightarrow \sum_{jk} \eps_{ijk} (z_j^* \d z_k^*) (z z^*)^{-2}  \\
\d & \leftrightarrow \dbar .
\end{align*}

\end{proof}

\subsection{dg model for vector fields}

Next we turn to an explicit model for dg model for vector fields on punctured affine space.
Algebraic sections of the tangent bundle $\T_{\A^n - \{0\}}$ is a sheaf of Lie algebras on punctured affine space and its derived global sections 
\begin{equation}
\R \Gamma(\A^n - \{0\}, \T_{\A^n-\{0\}})
\end{equation}
therefore carry the structure of a Lie algebra, up to homotopy.
We use the commutative dg algebra model $\sfA_{[n]}$ for punctured affine space recollected above to derive an explicit model for this as a dg Lie algebra.

The choice of a coordinate determines a trivialization of the tangent bundle on $\A^n - \{0\}$.
In particular
\begin{equation}
\R \Gamma(\A^n - \{0\}, \T_{\A^n-\{0\}}) \simeq \R \Gamma(\A^n - \{0\}, \cO_{\A^n - \{0\}}) \otimes \C^n
\end{equation}
where $\C^n$ is spanned by the vector fields $\frac{\del}{\del z_i}$, $i=1,\ldots,n$.

\begin{dfn}
The $n$-\defterm{dimensional Witt algebra} $\lie{witt}(n)$ is the dg Lie algebra whose underlying cochain complex is
\begin{equation}
\lie{witt}(n) = \sfA_{[n]} \otimes \C^n = \sfA_{[n]} \otimes \C \left\{\frac{\del}{\del z_i}\right\}_{i=1}^n
\end{equation}
and whose Lie bracket is
\begin{multline}
\left[\alpha_{I,i} (z,z^*) \d z^*_I \frac{\del}{\del z_i}, \beta_{J,j} (z,z^*) \d z^*_J \frac{\del}{\del z_j} \right] \\ = \left(\alpha_{I,i}(z,z^*) \frac{\del}{\del z_i} \beta_{J,j}(z,z^*)\frac{\del}{\del z_j} - (-1)^{|I|} \beta_{J,j}(z,z^*) \frac{\del}{\del z_j}\alpha(z,z^*) \frac{\del}{\del z_i}\right) \d z_I^* \d z_J^* .
\end{multline}
\end{dfn}


%\begin{align*}
%\d (\sum_i \lambda_i \mu_i) & = \sum_{ijk} \left(- \eps_{ijk} z_j \mu_k \mu_i + 2 \lambda_i z_i \omega \right) \\
%\end{align*}


\section{Chiral Liouville theory}

Let $\alpha \in \Omega^{1,\bu}(\C)$ be a spin $1$ field and consider the non-local action
\begin{equation}
\int_\C \alpha \dbar \del^{-1} \alpha .
\end{equation}

\section{Free field realization}

Let $\psi^\pm(z)$ be the fields of the free fermion system with OPE
\begin{equation}
\psi^+(z) \psi^-(w) \simeq \frac{1}{z-w} .
\end{equation}
These fields generate a simple vertex algebra that we denote by $F$.
We will call the vacuum vector $|0\> \in F$.

There is a family of stress tensors
\begin{equation}
T^\lambda (z) = (1-\lambda) \colon \del \psi^+(z) \psi^- (z) \colon  + \lambda \colon \del \psi^- (z)\psi^+(z) \colon ,
\end{equation}
which have central charge
\begin{equation}
c_\lambda = - 2 (6 \lambda^2 - 6 \lambda + 1) .
\end{equation}

This free fermion system exhibits the simplest free field realization.
Define the field
\begin{equation}
\alpha(z) \define \colon \psi^+(z) \psi^-(z) \colon 
\end{equation}
This is a free boson field of level $1$ and $\psi^\pm(z)$ has charge $\pm 1$ with respect to $\alpha(z)$.
This endows $F$ with the structure of a representation for the Heisenberg, or oscillator, algebra
\begin{equation}
0 \to \C K \to \Hat{\lie{s}} \to \C((t)) \to 0
\end{equation}
where $\alpha_j = t^j$ and the commutation relations are $[\alpha_i, \alpha_j] = i \delta_{i,-j}$.
Moreover, the stress tensor can be written in terms of this field as
\begin{equation}
T_\lambda(z) = \frac12 \colon \alpha(z)^2 \colon + \left(\frac12 - \lambda\right) \del \alpha (z) .
\end{equation}

The operator $\alpha_0$ defines a weight decomposition $F = \oplus_{m \in \Z} F^{(m)}$ where $\alpha_0$ acts on $F^{(m)}$ with eigenvalue $m$.
For $m > 0$ define the state
\begin{equation}
|m\> \define \psi^+_{(-m)} \cdots \psi_{(-2)}^+ \psi_{(-1)}^+ |0\> 
\end{equation}
and for $m < 0$ define the state
\begin{equation}
|m\> \define \psi^-_{(m)} \cdots \psi_{(-2)}^- \psi_{(-1)}^- |0\>  .
\end{equation}
Then $|m\> \in F^{(m)}$.
Further, these states exhibit the irreducibility of $F^{(m)}$ as a representation for the Heisenberg algebra $\Hat{\lie{s}}$.
Indeed, if $v \in F^{(m)}$ is any state with the property that $\alpha_j v = 0$ for all $j > 0$ then $v \in \C |m\>$.


\section{Lattices}
The target here is $\R^\ell / Q$.

Let $Q$ be the free abelian group of rank $\ell$ and consider the group algebra $\C[Q]$ with basis $e^{\alpha}, \alpha \in Q$ and multiplication $e^{\alpha} e^{\beta} = e^{\alpha + \beta}$.
We assume $Q$ is an integral lattice, meaning we have a symmetric bilinear form $(\cdot | \cdot) \colon Q \times Q \to \Z$.
We assume this bilinear form is non-degenerate.


Let $\lie{h} = \C \otimes_\Z Q \cong \C^\ell$ be the complexification of $Q$ and extend $(\cdot|\cdot)$ by linearity.
Denote by $\Hat{\lie{h}}$ the corresponding affine Heisenberg algebra with central parameter $K$.
The weight $\mu$ Verma module is
\begin{equation}
\til V (\vec{\mu}) = U(\Hat{\lie{h}}) \otimes_{U(\lie{h}[[t]] \oplus \C K)} \C_{\mu, 1}
\end{equation}
where $\C_{\vec{\mu},1}$ is the one-dimensional module where $t^0$ acts by the vector $\vec{\mu}$ and $K$ acts by $1$.
As a vector space
\begin{equation}
\til V (\vec{\mu}) \simeq S \left(\lie{h}^{<0}\right)
\end{equation}
where $\lie{h}^{<0} = \oplus_{j < 0} \lie{h} \otimes t^j$.
Write $S$ for this vector space.
In the case $\vec{\mu} = 0$, the vacuum module $S$ is equipped with the structure of a vertex algebra.

Let 
\begin{equation}
V_Q \define S \otimes \C[Q] .
\end{equation}

\section{Virasoro TFT}

Let $\Sigma$ be a Riemann surface and $S$ a one-dimensional smooth manifold.
We consider the local Lie algebra
\begin{equation}
\Omega^\bu(S) \hotimes \Omega^{0,\bu}(\Sigma, \T_{\Sigma}) 
\end{equation}
controlling deformations of the THF.

This theory has fields
\begin{align*}
\sfc & \in \Omega^\bu(S) \hotimes \Omega^{0,\bu}(\Sigma, \T_\Sigma)[1] \\
\sfb & \in \Omega^\bu(S) \hotimes \Omega^{0,\bu}(\Sigma, K_\Sigma^{\otimes 2})
\end{align*}
where the classical action is 
\begin{equation}
\int_{S \times \Sigma} \sfb (\d + \dbar) \sfc + \frac12 \int_{S \times \Sigma} \sfb [\sfc,\sfc] .
\end{equation}
The central charge should introduce a coupling like
\begin{equation}
\int_{S \times \C} J \sfc \del J \sfc .
\end{equation}

I claim that this factorization algebra is locally constant, so on $\R \times \C = \R^3$ we get an $\EE_3$ algebra.
The complex of local operators is equivalent to
\begin{equation}
C^\bu(\lie{vect}(1) \oplus F_{-2}[-1]) \simeq C^\bu(\lie{vect}(1) ; S^\bu \left(F^\vee_{-2}\right))
\end{equation}
where $F_{-2} = \Gamma(D, K^{\otimes 2})$.
In other words we have polynomials in $\del^\bu \sfc$ and $\del^\bu \sfb$.
Since $\sfb$ is spin $2$ the cohomology of this is the same as the cohomology of just the $\sfc$'s:
\begin{equation}
H^\bu(\lie{vect}(1)) = \C \oplus \C[-3] .
\end{equation}
This is hopeful, it's the same as the cohomology of local operators in Chern--Simons theory for $\lie{sl}(2)$.

I think that the $\EE_3$ algebra $C^\bu(\lie{sl}(2))$ underlying Chern--Simons for $\lie{sl}(2)$ is trivializable.
But it is nontrivial as a filtered $\EE_3$ algebra where the filtration is by symmetric degree.
Can we identify this filtration in this theory?



\section{Holomorphic 3d-3d}

Consider compactification of the free tensor multiplet along
\begin{equation}
C \times (\C^2 - \{0\})
\end{equation}
where $C$ is a Riemann surface.
To compactify we replace $\Omega^{0,\bu}(\C^2 - \{0\})$ by
\begin{equation}
\Omega^{\bu}(\R) \otimes \left(\C[w_1,w_2] \oplus (w_1 w_2)^{-1}\C[w_1^{-1},w_2^{-1}][-1] \right)
\end{equation}
Also, let $\cA^{i,\bu}(C \times \R)$ denote $\Omega^{i,\bu}(C) \otimes \Omega^\bu(\R)$ for $i=0,1$.

For the tensor, observe
\begin{equation}
\Omega^{2,\bu}(C \times (\C^2 - \{0\})) \simeq \Omega^{0,\bu}(C) \otimes \Omega^{2,\bu}(\C^2 -\{0\}) \oplus \Omega^{1,\bu}(C) \otimes \Omega^{1,\bu}(\C^2 - \{0\}) ,
\end{equation}
and 
\begin{equation}
\Omega^{3,\bu}(C \times (\C^2 - \{0\})) \simeq \Omega^{1,\bu}(C) \otimes \Omega^{2,\bu}(\C^2 - \{0\}) .
\end{equation}

So the complex of fields is
\begin{equation}
\begin{tikzcd}
\ul{-1} & \ul{0} & \ul{1} \\
\cA^{0,\bu} [w_1,w_2] \d^2 w \ar[r,"\del_C"] & \cA^{1,\bu}[w_1,w_2] \d^2 w & \\
\cA^{1,\bu}[w_1,w_2] \d w_i \ar[ur,"\del_w"] & & \\
& (w_1w_2)^{-1}\cA^{0,\bu} [w_1^{-1},w_2^{-1}] \d^2 w \ar[r,"\del_C"] & (w_1w_2)^{-1}\cA^{1,\bu}[w_1^{-1},w_2^{-1}] \d^2 w \\
& (w_1 w_2)^{-1}\cA^{1,\bu}[w_1^{-1},w_2^{-1}] \d w_i \ar[ur,"\del_w"] & 
\end{tikzcd}
\end{equation}

This looks to be quasi-isomorphic to
\begin{equation}
\begin{tikzcd}
\ul{-1} & \ul{0} & \ul{1} \\
\cA^{0,\bu} [w_1,w_2] \d^2 w & & \\
\cA^{1,\bu} \otimes (\C[w_1,w_2] \slash \C) & & \\
& (w_1w_2)^{-1}\cA^{0,\bu} [w_1^{-1},w_2^{-1}] \d^2 w & \\
& \cA^{1,\bu} \otimes \left( (w_1 w_2)^{-1} \C[w_1^{-1},w_2^{-1}] / (\C \cdot (w_1 w_2)^{-1}) \right)  & 
\end{tikzcd}
\end{equation}

The compactification of the hypermultiplet is
\begin{equation}
\begin{tikzcd}
\ul{-2} & \ul{-1} & \ul{0} & \ul{1} \\
\cA^{0,\bu}[w_1,w_2] &  (w_1w_2)^{-1}\cA^{0,\bu} [w_1^{-1},w_2^{-1}] \\
& & \cA^{1,\bu}[w_1,w_2] \d^2 w &  (w_1w_2)^{-1}\cA^{1,\bu} [w_1^{-1},w_2^{-1}] \d^2 w .
\end{tikzcd}
\end{equation}

It appears if we take $\C^\times$-invariants where $w_i$ has weight one, then the only thing that survives is from the hyper part. 
Namely
\begin{equation}
\begin{tikzcd}
\ul{-2} & \ul{-1} & \ul{0} & \ul{1} \\
\cA^{0,\bu} &  & & \\
& & &  (w_1w_2)^{-1}\cA^{1,\bu} \d^2 w .
\end{tikzcd}
\end{equation}
If we regrade, then I think we are looking OK; this is level zero limit of $\lie{gl}(1)$ Chern--Simons theory.
We should see the level from the shifted Poisson structure somehow.

Something is weird with the spin, I think the correct `twist' replaces the hyper above with 
\begin{align*}
\gamma & \in \Omega^{1,\bu}(\C) \otimes \Omega^{0,\bu}(\C^2 - \{0\}) \\
\beta & \in \Omega^{0,\bu}(C) \otimes \Omega^{2,\bu}(\C^2 - \{0\}) .
\end{align*}
Then the compactification looks like
\begin{equation}
\begin{tikzcd}
\ul{-2} & \ul{-1} & \ul{0} & \ul{1} \\
\cA^{0,\bu}[w_1,w_2] \d^2 w &  (w_1w_2)^{-1}\cA^{0,\bu} [w_1^{-1},w_2^{-1}] \d^2 w \\
& & \cA^{1,\bu}[w_1,w_2] &  (w_1w_2)^{-1}\cA^{1,\bu} [w_1^{-1},w_2^{-1}]  .
\end{tikzcd}
\end{equation}

Next, we want to understand what happens when we turn on the $\Omega$-background/superconformal deformation.

\cite{ButsonYoo}

\printbibliography

\end{document}














Let 
\begin{equation}
\sfR = \C[z_i,\lamba_j,\mu_k]_{i=1,2,3} / (1-z_i \lambda_i, 
\eeq
where $z_i, \lambda_j$ have degree zero for $i,j=1,2,3$ and $\mu_k$ is degree one for $k=1,2,3$.
In particular $z_i \mu_j = \mu_j z_i$ and $\mu_i \mu_j = - \mu_j \mu_i$.

In degree zero we have polynomials in $z_i,\lambda_i$ where we identify polynomials according to the relation $z_i \lambda_i =1$.
