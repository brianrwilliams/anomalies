\documentclass[11pt]{amsart}

\usepackage{../macros,amsaddr,physics}
\numberwithin{equation}{section}

\setcounter{tocdepth}{1}

%\linespread{1.2} %for editing
%\usepackage{mathpazo}




\begin{document}

\title{Response to referee report}

\maketitle

I thank the referee for their detailed reading of my manuscript and for the interesting and very helpful questions and
comments. 
Below, I have responded to each of these questions and comments.

\begin{itemize}
  \item I have added a remark after the definition of $\cL$-background to point to the familiar notion of "background
    fields".
    I have also added a paragraph in the introduction which more explicitly relates this work to the untwisted
    superconformal setting.
  \item We only consider theories which admit quantizations that are exact (or truncate) at one-loop in perturbation
    theory.
    For this reason, all of our anomalies necessarily occur at one-loop.
    I do not know of any theories which admit a one-loop symmetry of holomorphic vector fields but are  obstructed to
    having a higher-loop symmetry.
    It is an interesting question.
    I think to construct such an example one needs to consider more exotic theories that do not admit a strict action
    of holomorphic vector fields, but rather a homotopical one.
  \item There is no problem extending the results to non-rational $\lambda$ on flat space (which is all we consider
    here). But, we lose the geometric interpretation in this case. 
    I have added a paragraph at the end of section 3.2 detailing this.
  \item Proposition 5.2 proves that if we choose the bundle where the holomorphic quarks live correctly, then the
    theory has a quantum background for all holomorphic vector fields.
    Since holomorphic symplectic vector fields are automatically divergence-free, we see that symplectic vector fields
    will be a quantum symmetry for \textit{any} choice of $\lambda$ (hence for any choice of line bundle to twist the
    holomorphic quarks by).
    At the present time, I am unaware how to see the superconformal window at the level of the holomorphic twist.
    What I can prove is that the twist of any superconformal theory has a quantum background for all holomorphic
    vector fields.
    On the other hand, this is not a necessary condition; there are many supersymmetric theories (like QCD) which are
    non-conformal yet whose holomorphic twists admit a symmetry by holomorphic vector fields.
    I think what needs to be uncovered is how the process of RG flow intertwines with twisting.
    This would be a big breakthrough towards a holomorphic understanding of Seiberg duality.
 \item As indicated in the last bullet point, I know of many theories which are not conformal yet twist to a theory
   which admits a holomorphic vector field symmetry. 
  My conjectural interpretation is that $a^{hol}, c^{hol}$, and hence $a,c$, for these theories should be the ones
  associated to the superconformal theory that the original theory flows to (if such a flow exists).
  I am happy to put this in the draft if the referee thinks it is illuminating.
\item For the last typo suggestion, I believe that in equation (4.7) the space that $\beta$ lives is correct. 
  The factor of the canonical bundle is absorbed into the coefficients of the Dolbeault complex.
  \end{itemize}
\end{document}
